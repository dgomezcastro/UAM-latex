% LTeX: language=es-ES

\setcounter{part}{0}
\part{Bienvenido a \LaTeX}\label{sesion:1}
\firstslide

\frame{\tableofcontents}

\section{Primeros pasos}
\frame{\sectionpage}

\begin{frame}
	\frametitle{Overleaf}
	Una opción sencilla es usar \href{https://www.overleaf.com/}{Overleaf}
		\includegraphics[height=.05\textheight]{imagenes/overleaf} 
	\begin{figure}
		\centering
		\includegraphics[width=.9\textwidth]{imagenes/overleaf-screenshot}
		\caption{Interfaz de Overleaf}
	\end{figure}
	
\end{frame}

\section{¡Hola, Mundo! e ingredientes básicos}
    
\begin{frame}{El archivo mínimo}
    Los archivos de \LaTeX\ son archivos de texto (plano) con extension \texttt{.tex}. 

	La estructura es
    
\begin{block}{Código}
	\latexfile[highlightlines={}]{codigos/vacio.tex}
\end{block}

	Estos documentos se ``compilan'' para conseguir un documento ``con formato''.
\end{frame}

\begin{frame}
    \frametitle{El archivo ¡Hola, Mundo!}
    Creemos un archivo básico
    \begin{block}{Código [hola-mundo.tex]}
        \latexfile{codigos/hola-mundo.tex}
    \end{block}
\end{frame}

\begin{frame}[fragile]{Comentarios}
	Lo que venga después de \latexinline|%| no se procesará como comandos o contenido, 

	si no como comentarios.
	\begin{block}{Código [hola-mundo-con-comentario.tex]}
        \latexfile{codigos/hola-mundo-con-comentario.tex}
    \end{block}
\end{frame}

\begin{frame}[fragile]{Errores y avisos}
	¿Qué pasa si comentemos un error? 

	\begin{block}{Código}
        \begin{latexsource*}{highlightlines={3}}
\documentclass{article}

\begin{documet}
		
Hola, Mundo.
		
\end{document}
		\end{latexsource*}
    \end{block}

	Al compilar
	\begin{minted}{bash}
! LaTeX Error: Missing \begin{document}.
	\end{minted}

	Hay tres tipos de mensajes:
	\begin{itemize}
		\item {\color{red}\texttt{Error}} normalmente no permitirá la compilación total o parcial. 

		\item {\color{yellow}\texttt{Warning}} que solo avisan de un posible mal funcionamiento.
	
		\item {\color{blue}Info} son cuestiones menores, que conviene revisar.
	\end{itemize}
\end{frame}

\section{Estructura del documento y principales elementos}
\frame{\sectionpage}

\begin{frame}
\frametitle{Cabecera y tipo de documento}
En la cabecera introduciremos todo lo relativo a configuración
\begin{block}{Código}
	\latexfile[highlightlines={1-3}]{codigos/vacio.tex}
\end{block}
\end{frame}

\begin{frame}{Tipo de documento}
Aquí es donde decidimos que tipo de archivo latex queremos escribir, hay diferentes tipos de documentos
\begin{columns}
	\column{.4\textwidth}
	\texttt{<style>}:
	\begin{enumerate}
		\item<1> \texttt{article} \only<1>{Para artículos cortos. Acepta partes, secciones y subsecciones}
		\item<2> \texttt{book} \only<2>{Para archivos más extensos. Acepta partes, capítulos, secciones, subsecciones}
	\end{enumerate}
	\column{.5\textwidth}
	\begin{block}{Código}
		\latexfile[highlightlines={1}]{codigos/vacio.tex}
	\end{block}
\end{columns}
\end{frame}

\begin{frame}
    \frametitle{El cuerpo}
    A partir de aquí escribiremos el texto
    \begin{columns}
        \column{.4\textwidth}
        Todo lo que queramos escribir.
        \column{.5\textwidth}
        \begin{block}{Código}
			\latexfile[highlightlines={5,9}]{codigos/vacio.tex}
        \end{block}
    \end{columns}
\end{frame}

\begin{frame}[fragile]
    \frametitle{Comandos y variables}
    Una herramienta fundamental en la escritura con \LaTeX
    \begin{columns}
        \column{.4\textwidth}
        \begin{itemize}
             \item \texttt{<command>} Nombre del comando
             \item \texttt{<opt>} Argumento optativo.
             \item \texttt{<arg\#>} Argumento obligatorio
        \end{itemize}
        \column{.6\textwidth}
        \begin{block}{Código (llamada a comando)}
            \begin{latexsource}
\<command>[<opt>]{<arg1>}{<arg2>}
            \end{latexsource}

        \end{block}
    \end{columns}

	\bigskip 
	Por ejemplo,
	\begin{block}{Código}
	\begin{latexsource}
Hola, \textbf{Mundo}.
	\end{latexsource}
	\end{block}
\end{frame}


\begin{frame}[fragile]
    \frametitle{Entornos}
Los entornos funcionan como comandos, pero nos permiten introducir cantidades más largas de texto.
\begin{columns}
    \column{.4\textwidth}
    Algunos ejemplos son
    \begin{itemize}
        \item \texttt{document}: Es donde introducimos el documento
        \item \texttt{equation}: Para introducir ecuaciones numeradas
        \item \texttt{emph}: Para conseguir textos en cursiva.
    \end{itemize}
    \column{.5\textwidth}
    \begin{block}{Código}
        \begin{latexsource}
\begin{<env>}[<opt>]


\end{<env>}
        \end{latexsource}
    \end{block}
\end{columns}
\end{frame}

\begin{frame}{Los paquetes} 

Por defecto \LaTeX\ no incluye demasiados comandos ni entornos. Podemos añadir nuevas funcionalidades (comandos y entornos) incluyendo \textbf{paquetes}. 

Uno de los paquetes más usuales es el paquete matemático de la American Mathematical Society (AMS): \texttt{amsmath}.
\begin{columns} 
	\column{.4\textwidth}
\begin{block}{Código}
	\latexfile[highlightlines={3,6,8}]{codigos/ejemplo-amsmath.tex}
\end{block}

	\column{.5\textwidth} 
	\fbox{ 
		\includegraphics[width=\textwidth]{codigos/ejemplo-amsmath-preview.pdf}
	}
\end{columns}
\end{frame} 

\begin{frame}
	\frametitle{Creando comandos} 
	\begin{columns} 
		\column{.6\textwidth}
		\begin{block}{Código}
			\latexfile[highlightlines={3-5,8}] 
			{codigos/ejemplo-comandos.tex}
		\end{block}
		
		\column{.3\textwidth} 
		\fbox{ 
			\includegraphics[width=\textwidth]{codigos/ejemplo-comandos.pdf}
		}
	\end{columns}
	
\end{frame}
        

\begin{frame}[fragile]
    \frametitle{El fichero y compatibilidades}
    \framesubtitle{El paquete \texttt{inputenc}}
    Para mayor compatibilidad, especialmente entre sistemas operativos es recomendable guardar los archivos de \texttt{.tex} en formato UTF8. Esto nos permitirá poner acentos de manera sencilla.
    \begin{columns}
        \column{.4\textwidth}
        Para indicarle al compilar que hemos hecho eso escribimos.
        \column{.5\textwidth}
        \begin{block}{Código}
            \begin{latexsource*}{highlightlines=3}
\documentclass{<style>}   
            
\usepackage[utf8]{inputenc}
            
\begin{document}
            \end{latexsource*}
        \end{block}
    \end{columns}
\end{frame}

\begin{frame}[fragile]
    \frametitle{El paquete \texttt{babel}}
    Para que \LaTeX\ ponga todos los textos automáticos en castellano deberemos añadir el paquete \texttt{babel}
     \begin{columns}
         \column{.45\textwidth}
         Para indicarle al compilar que hemos hecho eso escribimos.
         \column{.45\textwidth}
         \begin{block}{Código}
             \small
             \begin{latexsource*}{highlightlines={3}}
\documentclass{<style>}   

\usepackage[spanish]{babel}

\begin{document}
             \end{latexsource*}
            \end{block}
        \end{columns}
\end{frame}



\begin{frame}[fragile]
    \frametitle{Aspecto de un primer documento}
    \begin{columns}
        \column{.5\textwidth}
        \begin{block}{Código [\texttt{basico.tex}]}
        \scriptsize
        \latexfile{codigos/basico.tex}
        \end{block}
        \column{.5\textwidth}
        \fbox{
        \includegraphics[scale=.25]{codigos/basico.pdf}
    }
    \end{columns}
\end{frame}

\begin{frame}[fragile]
	\frametitle{Ficheros modulares: \texttt{input}}
	Escribir un libro completo en un único archivo no es cómodo. Por eso \LaTeX\ permite escribir modularmente. 
	
	Podemos escribir en diferentes archivos \texttt{.tex}, y luego juntarlos en un principal.
	\begin{columns}
		\column{.5\textwidth}
		\begin{block}{Código [\texttt{modular.tex}]}
			\latexfile[highlightlines={3-4}]{codigos/modular.tex}
		\end{block}
		\begin{block}{Código [\texttt{modulo1.tex}]}
			\latexfile{codigos/modulo1.tex}
		\end{block}
		\begin{block}{Código [\texttt{modulo2.tex}]}
			\latexfile{codigos/modulo2.tex}
		\end{block}
		\column{.4\textwidth}
		\begin{figure}
			\centering 
			\fbox{\includegraphics[]{codigos/modular.pdf}}
			\caption{Resultado de compilar \texttt{modular.tex}}
		\end{figure} 
	\end{columns}
\end{frame}

\section{\LaTeX\ en local}
\frame{\sectionpage}

\begin{frame}[fragile]{Descargar e instalar \LaTeX} 
	Esta experiencia depende del sistema utilices: visita \href{https://www.latex-project.org/get/}{Latex project}
	\begin{itemize}
		\item Windows: \href{https://miktex.org/download}{MikTeX}
		\item Mac: \href{http://www.tug.org/mactex/}{MacTeX} (o via \texttt{homebrew})
		\item Linux: a través del gestor software nativo de terminal
		\begin{itemize}
			\item Debian/Ubuntu: \texttt{sudo apt-get install texlive-full} 
			\item RedHat/Fedora: \mintinline{bash}|yum install texlive-scheme-full|
			\item Suse: \texttt{zypper install texlive-latex}
			\item Arch: \texttt{pacman -S texlive-most}
			\item Otros: ¿en serio? ¿ninguno de los anteriores?. Te buscas la vida.
		\end{itemize}
	\end{itemize}
	
\end{frame}


\begin{frame}[fragile]{Compilando en terminal}
	
	Navegar hasta la carpeta y escribir en terminal 
	\begin{minted}{bash} 
$ pdflatex hola-mundo.tex
	\end{minted} 
	En archivos más complicados hay que ejecutar el código varias veces 
	
\end{frame}

\begin{frame}{Diferentes compiladores}
	Hay diferentes opciones

	\begin{enumerate}
		\item \texttt{pdflatex}. El más habitual. Genera un archivo \texttt{.pdf}
		\item \texttt{latex}. El más tradicional. Genera un archivo \texttt{.dvi}
		\item \texttt{XeLaTeX} y \texttt{LuaLaTeX} soportan tipografías Truetype and OpenType. 
		
		Se usan a veces para documentos muy visuales. 
	\end{enumerate}
	
\end{frame}

\begin{frame}{Editores locales}
	Un archivo \texttt{.tex} es archivo de texto ``plano''. Se puede editar con cualquier editor. 

	Algunos editores tienen funcionalidades avanzadas (compilación integrada, \texttt{synctex}, visor de pdf...), que simplifican el trabajo
	\begin{enumerate}
		\item Visual Studio Code. Con la extensión Latex-workshop.
	
		\item Latex Workshop
		
		\item Texstudio
	\end{enumerate}
\end{frame}

% \begin{frame}
% 	\frametitle{Descargar el editor}
% 	\framesubtitle{TeXstudio} 
	
% 	Para utilizar en local, recomiendo TeXstudio \href{https://www.texstudio.org}{TeXstudio}. 
% 	Es libre y gratuito\footnote{Los usuarios de Linux lo pueden descargar por terminal}. \\[3ex]
	
	
% 	Hay más opciones:
% 	\begin{enumerate}
% 		\item TeXShop
% 		\item TexMaker
% 		\item Gummy
% 		\item Atom (requiere alguna configuración)
% 		\item Emacs, Vim, etc... + compilación por terminal
% 	\end{enumerate}
	

% \end{frame}

% \begin{frame}
% 	\frametitle{TeXstudio}
	
% 		\begin{figure}
% 		\centering
% 		\includegraphics[width=.9\textwidth]{imagenes/texstudio}
% 		\caption{Interfaz de TeXstudio}
% 	\end{figure}
	
% \end{frame}

% \begin{frame}
% \frametitle{Compilando con TeXstudio}

% \begin{figure}
% 	\centering
% 	\includegraphics[width=.9\textwidth]{imagenes/texstudio-compilar}
% 	\caption{Compilar con TeXstudio}
% \end{figure}

% \end{frame}

\begin{frame}
	\frametitle{El fichero y compatibilidades}
	\small 
	Los ficheros de texto plano en principio soportan muy pocos caracteres (\href{https://en.wikipedia.org/wiki/ASCII}{ASCII}). \\

	Para aumentar el número de caracteres soportados (por ejemplos añadir letras con acentos) hay diferentes codificaciones: 
	
	UTF-8 (por defecto en Linux y Mac), ISO (por defecto en Windows), ... \\

	Para aprovechar esto usamos el paquete \texttt{inputenc} con la codificación correcta.
	\begin{figure}
		\includegraphics[width=.5\textwidth]{imagenes/texstudio-utf8}
	\end{figure}

	\textbf{Curiosidades sobre texto plano}: \href{https://www.youtube.com/watch?v=4mRxIgu9R70}{\emph{Plain Text} by Dylan Beattie en Youtube}
\end{frame}