\section{Markdown}

\begin{frame}[fragile]{Markdown: archivos \texttt{.md}}
    \small
    \url{https://en.wikipedia.org/wiki/Markdown}
    \begin{quote}
        Markdown is a lightweight markup language for creating formatted text using a plain-text editor. John Gruber created Markdown in 2004 as an easy-to-read markup language.
    \end{quote}

    \pause

    \url{https://daringfireball.net/projects/markdown/}:
    \begin{quote}
        Markdown is a text-to-HTML conversion tool for web writers. Markdown allows you to write using an easy-to-read, easy-to-write plain text format, then convert it to structurally valid XHTML (or HTML).

        Thus, “Markdown” is two things: (1) a plain text formatting syntax; and (2) a software tool, written in Perl, that converts the plain text formatting to HTML. See the \href{https://daringfireball.net/projects/markdown/syntax}{Syntax} page for details pertaining to Markdown’s formatting syntax. You can try it out, right now, using \href{https://daringfireball.net/projects/markdown/dingus}{the online Dingus}.
    \end{quote}

    \bigskip

    Por ejemplo, los \texttt{README.md} en \url{github.com}: \href{https://docs.github.com/en/get-started/writing-on-github/working-with-advanced-formatting/writing-mathematical-expressions}{link}.
\end{frame}

\begin{frame}[fragile]{\texttt{markdown}}
    \small

    Markdown utiliza algunas notaciones sencillas
    \begin{enumerate}
        \item
              \mintinline{markdown}|# 〈título〉| índica el inicio de la sección 〈título〉. Similar para subsección y sub-subsección.

        \item
              \mintinline{markdown}|-| al comienzo de línea indica que se trata de una lista no numerada, y  \mint{markdown}|1.| de lista numerada.

        \item
              \mintinline{markdown}|*〈texto〉*| indica texto en cursiva, \mintinline{markdown}|**| indica negrita, y \mintinline{markdown}|***| indica negrita y cursiva

        \item
              \mintinline{markdown}|[Duck Duck Go](https://duckduckgo.com)| crea un link

        \item
              \mintinline{markdown}|![〈texto〉](〈dirección〉)|
              inserta la imagen o video en 〈dirección〉utilizando 〈texto〉como texto alternativo.

        \item
              \mintinline{markdown}|$a=1$| es una ecuación en línea, y \mintinline{markdown}|$$a=1$$| una ecuación presentada
    \end{enumerate}
    \bigskip

    Más ejemplos en:
    \url{https://www.markdownguide.org/basic-syntax/}

    Probar en:
    \url{https://upmath.me/}
\end{frame}

\begin{frame}[fragile]{\texttt{pandoc}: de \texttt{.md} a \texttt{.tex}}
    \small

    \url{https://pandoc.org/}
    \begin{quote}
        you need to convert files from one markup format into another, pandoc is your swiss-army knife.
    \end{quote}
    \begin{block}{Archivo \texttt{ejemplo.md}}
        \inputminted{markdown}{codigos/ejemplo.md}
    \end{block}
    Mediante

    \mintinline{sh}|$ pandoc ejemplo.md -o ejemplo.tex|

        obtenemos
        \begin{block}{Archivo \texttt{ejemplo.tex}}
            \inputminted{latex}{codigos/ejemplo.tex}
        \end{block}

        Podemos hacer el \texttt{.tex} tenga cabecera indicando
        \mintinline{sh}|$ pandoc ejemplo.md --standalone -o ejemplo_standalone.tex|
\end{frame}