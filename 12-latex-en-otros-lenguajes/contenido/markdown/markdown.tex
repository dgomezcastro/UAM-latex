\section{Markdown}

\begin{frame}[fragile]{Markdown}
    \url{https://en.wikipedia.org/wiki/Markdown}
    \begin{quote}
        Markdown is a lightweight markup language for creating formatted text using a plain-text editor. John Gruber created Markdown in 2004 as an easy-to-read markup language.
    \end{quote}

    \pause

    \url{https://daringfireball.net/projects/markdown/}:
    \begin{quote}
        Markdown is a text-to-HTML conversion tool for web writers. Markdown allows you to write using an easy-to-read, easy-to-write plain text format, then convert it to structurally valid XHTML (or HTML).

        Thus, “Markdown” is two things: (1) a plain text formatting syntax; and (2) a software tool, written in Perl, that converts the plain text formatting to HTML. See the \href{https://daringfireball.net/projects/markdown/syntax}{Syntax} page for details pertaining to Markdown’s formatting syntax. You can try it out, right now, using \href{https://daringfireball.net/projects/markdown/dingus}{the online Dingus}.
    \end{quote}

    \bigskip 

    Ejemplo de uso:
    \url{https://upmath.me/}
\end{frame}