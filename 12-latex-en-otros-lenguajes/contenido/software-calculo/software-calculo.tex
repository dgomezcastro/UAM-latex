\section{LaTeX en software científico}

\begin{frame}[fragile]{\texttt{jupyter}: libretas para \texttt{julia}, \texttt{python}, \texttt{R} (\texttt{.ipynb})}
    \texttt{jupyter} admite celdas de código y celdas de texto \texttt{markdown}.

    \bigskip

    En ellas, se puede incluir \LaTeX{} (\href{https://jupyter-notebook.readthedocs.io/en/stable/examples/Notebook/Working%20With%20Markdown%20Cells.html}{documentación})

    \bigskip

    La herramienta \texttt{nbconvert} permite convertir \texttt{.ipynb} en \texttt{.tex} y \texttt{.html} (con MathJax)
\end{frame}


\begin{frame}[fragile]{Otros programas}
    \begin{itemize}
        \item \href{https://www.mathworks.com/help/symbolic/sym.latex.html}{MATLAB}
        \item \href{https://doc.sagemath.org/html/en/tutorial/latex.html}{Sage}
        \item \href{https://reference.wolfram.com/language/workflow/GenerateTeXWithTheWolframLanguage.html}{Mathematica}
        \item \href{https://maplesoft.com/support/help/maple/view.aspx?path=updates%2fMaple2021%2fLaTeX}{Maple}
        \item Julia: soporta caracteres unicode. Además \href{https://github.com/JuliaStrings/LaTeXStrings.jl}{LaTeXString.jl}
    \end{itemize}
\end{frame}
