\section{quarto}

\begin{frame}[fragile]{\texttt{quarto}}
    The wikipedia article (still in draft)
    \begin{quote}
        Quarto is a free and open-source scientific and technical publishing system developed by Posit PBC, the developer of RStudio and Tidyverse.

        Quarto converts a Quarto document into HTML, PDF, or other document formats, using Pandoc as the core alongside other software such as Jupyter.
    \end{quote}

    La sintaxis de \texttt{quarto} es una extensión de \texttt{markdown}.

    Los archivos utilizan la extensión \texttt{.qmd}
\end{frame}

\begin{frame}[fragile]{Primer ejemplo}
    \small

    \begin{block}{Archivo \texttt{basico.qmd}}
        \inputminted{markdown}{codigos/basico.qmd}
    \end{block}

    Compilando por terminal
    \mint{sh}|$ quarto render basico.qmd|

    crea \texttt{basico.html} muy básico

    \bigskip

    Podemos especificar cómo queremos que lo exporte:

    \begin{itemize}
        \item
              \LaTeX: \mint{sh}|quarto render basico.qmd --to latex| genera \texttt{basico.tex}
        \item
              \texttt{Reveal.Js}: \mint{sh}|quarto render basico.qmd --to revealjs -o basico_revealjs.html|
              genera \texttt{basico\_revealjs.html}
    \end{itemize}
\end{frame}

\begin{frame}[fragile]{Más sobre \texttt{quarto}}
    Algunos comentarios:
    \begin{itemize}
        \item Hay muchísimas opciones de configuración, en las que no entraremos: \url{https://quarto.org/}
        \item
              Es recomendable usar \texttt{quarto} con un IDE como \texttt{Visual Studio Code}, por las extensiones.
    \end{itemize}
\end{frame}

\begin{frame}[fragile]{Ejemplo}
    \url{https://gomezcastro.xyz/courses/MNA/01-FV-Transporte/slides.html}
\end{frame}