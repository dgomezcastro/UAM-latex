% LTeX: language=es-ES

\section{Cuestiones prácticas}

\begin{frame}{El profesor}
    \centering
	\href{http://gomezcastro.xyz}{gomezcastro.xyz}
\end{frame}

\begin{frame}[fragile,c]{Guía docente}
    \begin{center}
    \href{https://secretaria-virtual.uam.es/doa/consultaPublica/look%5bconpub%5dBuscarPubGuiaDocAs?entradaPublica=true&idiomaPais=es.ES&_centro=104&_planEstudio=592}{Web de la universidad}
    \end{center}
\end{frame}

\begin{frame}{Procedimiento de evaluación}
    \textbf{Texto oficial}

    \bigskip 

	El 80\% de la calificación del curso se obtendrá de las entregas semanales.

    \bigskip 

    El 20\% restante, se obtendrá del trabajo final.

    \bigskip

    Los requisitos mínimos para aprobar la asignatura son haber superado (calificación al menos 5) el 60\% de las tareas semanales, y haber entregado el trabajo final.

    \bigskip 

    \hrule 

    \bigskip

    \textbf{Adicional}

    Algunas semanas las entregas serán \textit{optativas}. 

    No contarán para el 60\% obligatorio, y su resultado servirá sólo para subir la calificación de curso.

\end{frame}

\begin{frame}{Bibliografía}
    \begin{itemize}
        \item T. Oetiker \textit{et al}. \href{https://tobi.oetiker.ch/lshort/lshort.pdf}{The Not So Short Introduction to \LaTeX}. Siempre en desarrollo.
        \item \url{https://tex.stackexchange.com}
        \item Documentación de los paquetes en \url{https://ctan.org}
    \end{itemize}
    Otros
    \begin{itemize}
        \item Tomás Bautista \textit{et al}. Una descripción de \LaTeXe. Documento electrónico.
        \item Bernardo Cascales Salinas \textit{et al}. El libro de LaTeX. Pearson, Prentice Hall, 2003.
        \item Leslie Lamport. LaTeX - A document preparation system. User’s guide and manual references. Addison-Wesley, 2nd ed. 1994.
    \end{itemize}
    Algunas opiones sobre escritura científica y en \LaTeX 
    \begin{itemize}
        \item Nicholas J. Higham. Handbook of Writing for the Mathematical Sciences, Third Edition. SIAM. 2020
    \end{itemize}
\end{frame}