% LTeX: language=es-ES

\section{Estructura del documento y principales elementos}
\frame{\sectionpage}

\begin{frame}
\frametitle{Cabecera y tipo de documento}
En la cabecera introduciremos todo lo relativo a configuración
\begin{block}{Código}
	\latexfile[highlightlines={1-3}]{codigos/vacio.tex}
\end{block}
\end{frame}

\begin{frame}{Tipo de documento}
Aquí es donde decidimos que tipo de archivo latex queremos escribir, hay diferentes tipos de documentos
\begin{columns}
	\column{.4\textwidth}
	\texttt{<style>}:
	\begin{enumerate}
		\item<1> \texttt{article} \only<1>{Para artículos cortos. Acepta partes, secciones y subsecciones}
		\item<2> \texttt{book} \only<2>{Para archivos más extensos. Acepta partes, capítulos, secciones, subsecciones}
	\end{enumerate}
	\column{.5\textwidth}
	\begin{block}{Código}
		\latexfile[highlightlines={1}]{codigos/vacio.tex}
	\end{block}
\end{columns}
\end{frame}

\begin{frame}
    \frametitle{El cuerpo}
    A partir de aquí escribiremos el texto
    \begin{columns}
        \column{.4\textwidth}
        Todo lo que queramos escribir.
        \column{.5\textwidth}
        \begin{block}{Código}
			\latexfile[highlightlines={5,9}]{codigos/vacio.tex}
        \end{block}
    \end{columns}
\end{frame}

\begin{frame}[fragile]
    \frametitle{Comandos y variables}
    Una herramienta fundamental en la escritura con \LaTeX
    \begin{columns}
        \column{.4\textwidth}
        \begin{itemize}
             \item \texttt{<command>} Nombre del comando
             \item \texttt{<opt>} Argumento optativo.
             \item \texttt{<arg\#>} Argumento obligatorio
        \end{itemize}
        \column{.6\textwidth}
        \begin{block}{Código (llamada a comando)}
            \begin{latexsource}
\<command>[<opt>]{<arg1>}{<arg2>}
            \end{latexsource}

        \end{block}
    \end{columns}

	\bigskip 
	Por ejemplo,
	\begin{block}{Código}
	\begin{latexsource}
Hola, \textbf{Mundo}.
	\end{latexsource}
	\end{block}
\end{frame}


\begin{frame}[fragile]
    \frametitle{Entornos}
Los entornos funcionan como comandos, pero nos permiten introducir cantidades más largas de texto.
\begin{columns}
    \column{.4\textwidth}
    Algunos ejemplos son
    \begin{itemize}
        \item \texttt{document}: Es donde introducimos el documento
        \item \texttt{equation}: Para introducir ecuaciones numeradas
        \item \texttt{emph}: Para conseguir textos en cursiva.
    \end{itemize}
    \column{.5\textwidth}
    \begin{block}{Código}
        \begin{latexsource}
\begin{<env>}[<opt>]


\end{<env>}
        \end{latexsource}
    \end{block}
\end{columns}
\end{frame}

\begin{frame}{Los paquetes} 

Por defecto \LaTeX\ no incluye demasiados comandos ni entornos. Podemos añadir nuevas funcionalidades (comandos y entornos) incluyendo \textbf{paquetes}. 

Uno de los paquetes más usuales es el paquete matemático de la American Mathematical Society (AMS): \texttt{amsmath}.
\begin{columns} 
	\column{.4\textwidth}
\begin{block}{Código}
	\latexfile[highlightlines={3,6,8}]{codigos/ejemplo-amsmath.tex}
\end{block}

	\column{.5\textwidth} 
	\fbox{ 
		\includegraphics[width=\textwidth]{codigos/ejemplo-amsmath-preview.pdf}
	}
\end{columns}
\end{frame} 


        

\begin{frame}[fragile]
    \frametitle{El fichero y compatibilidades}
    \framesubtitle{El paquete \texttt{inputenc}}
    Para mayor compatibilidad, especialmente entre sistemas operativos es recomendable guardar los archivos de \texttt{.tex} en formato UTF8. Esto nos permitirá poner acentos de manera sencilla.
    \begin{columns}
        \column{.4\textwidth}
        Para indicarle al compilar que hemos hecho eso escribimos.
        \column{.5\textwidth}
        \begin{block}{Código}
            \begin{latexsource*}{highlightlines=3}
\documentclass{<style>}   
            
\usepackage[utf8]{inputenc}
            
\begin{document}
            \end{latexsource*}
        \end{block}
    \end{columns}
\end{frame}

\begin{frame}[fragile]
    \frametitle{El paquete \texttt{babel}}
    Para que \LaTeX\ ponga todos los textos automáticos en castellano deberemos añadir el paquete \texttt{babel}
     \begin{columns}
         \column{.45\textwidth}
         Para indicarle al compilar que hemos hecho eso escribimos.
         \column{.45\textwidth}
         \begin{block}{Código}
             \small
             \begin{latexsource*}{highlightlines={3}}
\documentclass{<style>}   

\usepackage[spanish]{babel}

\begin{document}
             \end{latexsource*}
            \end{block}
        \end{columns}
\end{frame}



\begin{frame}[fragile]
    \frametitle{Aspecto de un primer documento}
    \begin{columns}
        \column{.5\textwidth}
        \begin{block}{Código [\texttt{basico.tex}]}
        \scriptsize
        \latexfile{codigos/basico.tex}
        \end{block}
        \column{.5\textwidth}
        \fbox{
        \includegraphics[scale=.25]{codigos/basico.pdf}
    }
    \end{columns}
\end{frame}

\begin{frame}[fragile]
	\frametitle{Ficheros modulares: \texttt{input}}
	Escribir un libro completo en un único archivo no es cómodo. Por eso \LaTeX\ permite escribir modularmente. 
	
	Podemos escribir en diferentes archivos \texttt{.tex}, y luego juntarlos en un principal.
	\begin{columns}
		\column{.5\textwidth}
		\begin{block}{Código [\texttt{modular.tex}]}
			\latexfile[highlightlines={3-4}]{codigos/modular.tex}
		\end{block}
		\begin{block}{Código [\texttt{modulo1.tex}]}
			\latexfile{codigos/modulo1.tex}
		\end{block}
		\begin{block}{Código [\texttt{modulo2.tex}]}
			\latexfile{codigos/modulo2.tex}
		\end{block}
		\column{.4\textwidth}
		\begin{figure}
			\centering 
			\fbox{\includegraphics[]{codigos/modular.pdf}}
			\caption{Resultado de compilar \texttt{modular.tex}}
		\end{figure} 
	\end{columns}
\end{frame}