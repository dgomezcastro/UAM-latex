% LTeX: language=es-ES

\section{\LaTeX\ en local}


\begin{frame}[fragile]{Descargar e instalar \LaTeX} 
	Esta experiencia depende del sistema utilices: visita \href{https://www.latex-project.org/get/}{Latex project}
	\begin{itemize}
		\item Windows: \href{https://miktex.org/download}{MikTeX}
		\item Mac: \href{http://www.tug.org/mactex/}{MacTeX} (o via \texttt{homebrew})
		\item Linux: a través del gestor software nativo de terminal
		\begin{itemize}
			\item Debian/Ubuntu: \texttt{sudo apt-get install texlive-full} 
			\item RedHat/Fedora: \mintinline{bash}|yum install texlive-scheme-full|
			\item Suse: \texttt{zypper install texlive-latex}
			\item Arch: \texttt{pacman -S texlive-most}
			\item Otros: ¿en serio? ¿ninguno de los anteriores?. Te buscas la vida.
		\end{itemize}
	\end{itemize}
	
\end{frame}


\begin{frame}[fragile]{Compilando en terminal}
	
	Navegar hasta la carpeta y escribir en terminal 
	\begin{minted}{bash} 
$ pdflatex hola-mundo.tex
	\end{minted} 
	En archivos más complicados hay que ejecutar el código varias veces 
	
\end{frame}

\begin{frame}{Diferentes compiladores}
	Hay diferentes opciones

	\begin{enumerate}
		\item \texttt{pdflatex}. El más habitual. Genera un archivo \texttt{.pdf}
		\item \texttt{latex}. El más tradicional. Genera un archivo \texttt{.dvi}
		\item \texttt{XeLaTeX} y \texttt{LuaLaTeX} soportan tipografías Truetype and OpenType. 
		
		Se usan a veces para documentos muy visuales. 
	\end{enumerate}
	
\end{frame}

\begin{frame}{Editores locales}
	Un archivo \texttt{.tex} es archivo de texto ``plano''. Se puede editar con cualquier editor. 

	Algunos editores tienen funcionalidades avanzadas (compilación integrada, \texttt{synctex}, visor de pdf...), que simplifican el trabajo
	\begin{enumerate}
		\item Visual Studio Code. Con la extensión Latex-workshop.
	
		\item Latex Workshop
		
		\item Texstudio
	\end{enumerate}
\end{frame}

% \begin{frame}
% 	\frametitle{Descargar el editor}
% 	\framesubtitle{TeXstudio} 
	
% 	Para utilizar en local, recomiendo TeXstudio \href{https://www.texstudio.org}{TeXstudio}. 
% 	Es libre y gratuito\footnote{Los usuarios de Linux lo pueden descargar por terminal}. \\[3ex]
	
	
% 	Hay más opciones:
% 	\begin{enumerate}
% 		\item TeXShop
% 		\item TexMaker
% 		\item Gummy
% 		\item Atom (requiere alguna configuración)
% 		\item Emacs, Vim, etc... + compilación por terminal
% 	\end{enumerate}
	

% \end{frame}

% \begin{frame}
% 	\frametitle{TeXstudio}
	
% 		\begin{figure}
% 		\centering
% 		\includegraphics[width=.9\textwidth]{imagenes/texstudio}
% 		\caption{Interfaz de TeXstudio}
% 	\end{figure}
	
% \end{frame}

% \begin{frame}
% \frametitle{Compilando con TeXstudio}

% \begin{figure}
% 	\centering
% 	\includegraphics[width=.9\textwidth]{imagenes/texstudio-compilar}
% 	\caption{Compilar con TeXstudio}
% \end{figure}

% \end{frame}

