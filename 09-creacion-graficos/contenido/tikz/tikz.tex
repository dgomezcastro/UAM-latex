\section{Dibujo básico} 

\begin{frame}
\frametitle{El paquete Ti$k$z}
	El paquete \texttt{tikz} nos permite general dibujos 
	\begin{columns} 
		\column{.5\textwidth}
		\begin{block}{Código} 
			\latexfile[fontsize=\footnotesize]{codigos/ejemplo-tikz.tex}
		\end{block} 
		\column{.4\textwidth}
		\begin{figure} 
			\fbox{\includegraphics[width=\textwidth]{codigos/ejemplo-tikz.pdf}}
			\caption{Resultado de compilar}
		\end{figure} 
	\end{columns}
\end{frame}

\begin{frame}{Dibujo libre}
	Existen diferentes tipos de líneas y figuras, las opciones son ilimitadas
	\begin{columns} 
		\column{.5\textwidth}
		\begin{block}{Código} 
			\latexfile[fontsize=\footnotesize]{codigos/ejemplo-tikz-2.tex}
		\end{block} 
		\column{.4\textwidth}
			\fbox{\begin{tikzpicture}
\draw[->] (-1.5,0) -- (1.5,0);
\draw[dashed] (0,-1.5) -- (0,1.5);
\draw[green] (0,0) circle (1cm);
\end{tikzpicture}}
	\end{columns}

	\vspace{3ex} 

	Bellos ejemplos se pueden encontrar en \url{http://www.texample.net/tikz/examples/}
\end{frame}

\begin{frame}{Representación de grafos} 
	\begin{block}{Código} 
		\latexfile[fontsize=\tiny]{codigos/ejemplo-tikz-graph.tex}
	\end{block} 

	\begin{figure} 
		\fbox{\includegraphics[height=.2\textheight]{codigos/ejemplo-tikz-graph.pdf}}
		\caption{Resultado de compilar}
	\end{figure} 
\end{frame}

\begin{frame}{Diagrama de flujo}
	\centering
	\includegraphics[height=.8\textheight]{codigos/diagram.pdf}
\end{frame}

\begin{frame}{Diagrama de flujo}
	\tiny
	\begin{block}{Código} 
		\begin{columns}[t] 
		\column{.45\textwidth}
			\latexfile[lastline=15]{codigos/diagram.tex}
			\column{.45\textwidth}
			\latexfile[firstline=16]{codigos/diagram.tex}

		\end{columns}
	\end{block} 
\end{frame}


\begin{frame}{Representación de curvas} 
		\begin{block}{Código} 
			\latexfile[fontsize=\tiny]{codigos/ejemplo-tikz-curva.tex}
		\end{block} 
		\begin{figure} 
			\centering
			\fbox{\includegraphics[width=.35\textwidth]{codigos/ejemplo-tikz-curva.pdf}}
		\end{figure}
\end{frame}

\begin{frame}{Desvaríos excesivos}
	\href{https://texample.net/tikz/examples/poincare/}
	{\includegraphics[width=.9\textwidth]{codigos/poincare.pdf}}
\end{frame}

\begin{frame}{Desvaríos II}
	\centering
	\href{https://texample.net/tikz/examples/totoro/}
	{\includegraphics[height=.8\textheight]{imagenes/totoro.png}}
\end{frame}

\section{Algunos paquete útiles}

\begin{frame}{\texttt{pgfplots}: representación de funciones} 
	\latexinputandpdf{codigos/ejemplo-tikz-curva-2}
\end{frame}


\begin{frame}{\texttt{pgfplots}: {Representación de datos}}
	\begin{columns} 
		\column{.55\textwidth}
		\begin{block}{Código} 
			\latexfile[fontsize=\footnotesize]{codigos/ejemplo-pgfplots-pointwise-data.tex}
		\end{block} 
		\vspace{-2ex}
		\begin{block}{Código [data.csv]} 
			\latexfile[fontsize=\tiny]{codigos/data.csv}
		\end{block} 
		\column{.4\textwidth}
		\begin{figure} 
			\fbox{\includegraphics[width=\textwidth]{codigos/ejemplo-pgfplots-pointwise-data.pdf}}
		\end{figure} 
	\end{columns}
\end{frame}

\begin{frame}[fragile]{\texttt{pgfplots}}
	Un manual razonablemente completo:
	\url{https://www.overleaf.com/learn/latex/Pgfplots_package}
	
	\bigskip

	Una buena lista de ejemplos del manual:
	\url{http://pgfplots.sourceforge.net/gallery.html}

\end{frame}

\begin{frame}{\texttt{mdframed}}
	\small
	\texttt{mdframed} es un paquete independiente, pero bien conectado con tikz (y pstricks)
	\latexinputandpdf{codigos/mdframed-1}

	\bigskip

	\href{https://ctan.javinator9889.com/macros/latex/contrib/mdframed/mdframed.pdf}{Link a documentación}
\end{frame}

\begin{frame}{\texttt{mdframed}}
	\tiny
	\latexinputandpdf{codigos/mdframed-2}
\end{frame}

\begin{frame}[fragile]{\href{https://ctan.org/pkg/tikz-cd}{\texttt{tikz-cd}}}
	\tiny
	\latexinputandpdf{codigos/tikz-cd}
\end{frame}

\begin{frame}[fragile]{\href{https://ctan.org/pkg/annotate-equations}{\texttt{annotate-equations}}}
	\tiny
	\latexinputandpdf{codigos/tikz-annotate}
\end{frame}

\begin{frame}[fragile]{\href{https://ctan.org/pkg/tikzducks}{\texttt{ducks}}}
	\small
	\latexinputandpdf{codigos/tikz-ducks}
\end{frame}