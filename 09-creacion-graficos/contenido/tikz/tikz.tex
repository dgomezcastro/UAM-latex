\section{Dibujo libre} 

\begin{frame}
\frametitle{El paquete Ti$k$z}
El paquete \texttt{tikz} nos permite general dibujos 
	\begin{columns} 
		\column{.5\textwidth}
		\begin{block}{Código} 
			\latexfile[fontsize=\footnotesize]{codigos/ejemplo-tikz.tex}
		\end{block} 
		\column{.4\textwidth}
		\begin{figure} 
			\fbox{\includegraphics[width=\textwidth]{codigos/ejemplo-tikz.pdf}}
			\caption{Resultado de compilar}
		\end{figure} 
	\end{columns}
\end{frame}

\begin{frame}
\frametitle{Dibujo libre}
Existen diferentes tipos de líneas y figuras, las opciones son ilimitadas
\begin{columns} 
	\column{.5\textwidth}
	\begin{block}{Código} 
		\latexfile[fontsize=\footnotesize]{codigos/ejemplo-tikz-2.tex}
	\end{block} 
	\column{.4\textwidth}
		\fbox{\begin{tikzpicture}
\draw[->] (-1.5,0) -- (1.5,0);
\draw[dashed] (0,-1.5) -- (0,1.5);
\draw[green] (0,0) circle (1cm);
\end{tikzpicture}}
\end{columns}

\vspace{3ex} 

Bellos ejemplos se pueden encontrar en \url{http://www.texample.net/tikz/examples/}
\end{frame}

\begin{frame}
	\frametitle{Representación de grafos} 
	\begin{block}{Código} 
		\latexfile[fontsize=\tiny]{codigos/ejemplo-tikz-graph.tex}
	\end{block} 

	\begin{figure} 
		\fbox{\includegraphics[height=.2\textheight]{codigos/ejemplo-tikz-graph.pdf}}
		\caption{Resultado de compilar}
	\end{figure} 
\end{frame}

\begin{frame}
\frametitle{Diagramas de flujo}
\includegraphics[width=\textwidth]{imagenes/18-PONE-flow}
\end{frame}

\begin{frame}
	\frametitle{Desvaríos excesivos}
	\includegraphics[width=.9\textwidth]{codigos/poincare.pdf}
\end{frame}

\section{Representación de funciones} 

\begin{frame}
	\frametitle{Representación de curvas} 
		\begin{block}{Código} 
			\latexfile[fontsize=\tiny]{codigos/ejemplo-tikz-curva.tex}
		\end{block} 
		\begin{figure} 
			\centering
			\fbox{\includegraphics[width=.35\textwidth]{codigos/ejemplo-tikz-curva.pdf}}
		\end{figure}
\end{frame}

\begin{frame}
\frametitle{Representación de funciones: \texttt{pgfplots}} 
	\begin{columns} 
	\column{.55\textwidth}
	\begin{block}{Código} 
		\latexfile[fontsize=\footnotesize]{codigos/ejemplo-tikz-curva-2.tex}
	\end{block} 
	\column{.4\textwidth}
	\begin{figure} 
		\fbox{\includegraphics[width=\textwidth]{codigos/ejemplo-tikz-curva-2.pdf}}
		\caption{Resultado de compilar}
	\end{figure} 
	\end{columns}

	\vspace{3ex} 
	
	Una buena lista de ejemplos del manual:
	\url{http://pgfplots.sourceforge.net/gallery.html}
\end{frame}


\section{Representación de datos} 

\begin{frame}
	\frametitle{{Representación de datos}}
	\begin{columns} 
		\column{.55\textwidth}
		\begin{block}{Código} 
			\latexfile[fontsize=\footnotesize]{codigos/ejemplo-pgfplots-pointwise-data.tex}
		\end{block} 
	
		\begin{block}{Código [data.csv]} 
			\latexfile[fontsize=\tiny]{codigos/data.csv}
		\end{block} 
		\column{.4\textwidth}
		\begin{figure} 
			\fbox{\includegraphics[width=\textwidth]{codigos/ejemplo-pgfplots-pointwise-data.pdf}}
			\caption{Resultado de compilar}
		\end{figure} 
	\end{columns}
\end{frame}