\subsection{Sincronizando \texttt{.pdf} y \texttt{.tex}: \texttt{synctex}}

\begin{frame}[fragile]{Sincronizando \texttt{.pdf} y \texttt{.tex}: \texttt{synctex}}
    \url{https://ctan.org/pkg/synctex-parser}
    \begin{quote}
    Since 2008, the Synchronization TEXnology named SyncTEX is a feature of TEX engines. It allows to synchronize between input and output, which means to navigate from the source document to the typeset material and vice versa. Here is how it works: during the typesetting process of foo.tex, the TEX engines writes some geometrical information in an auxiliary file named foo.synctex, this information is then used by editors or viewers to navigate between input and output.
    \end{quote}

    Si compilamos con
    \begin{minted}{console}
pdflatex --synctex=1 ...
    \end{minted}
    Se genera un archivo \texttt{.synctex.gz} que permite a los editores que lo soportan sincronizar en el \texttt{.pdf} y el \texttt{.tex}.
\end{frame}