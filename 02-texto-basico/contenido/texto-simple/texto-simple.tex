% LTeX: language=es-ES

\section{Simplemente texto}
\frame{\sectionpage}

\begin{frame}{Escribimos}
    % TODO
\end{frame}

\begin{frame}[fragile]{Negrita, cursiva, ... }
    \begin{table} 
		\centering
	\begin{tabular}{l|l}
        \latexinline{Hola} & Hola \\
        \latexinline{\textbf{Hola}} & \textbf{Hola} \\
        \latexinline{\textit{Hola}} & \textit{Hola} \\
        \latexinline{\texttt{Hola}} & \texttt{Hola} \\
        \latexinline{\textsc{Hola}} & \textsc{Hola} \\
        \latexinline{\underline{Hola}} & \underline{Hola} \\
	\end{tabular}
	\end{table} 
    Otros paquetes incluyen más funcionalidades: \texttt{ulem}, \texttt{sout} permiten tachar texto, ... 
\end{frame}

\begin{frame}[fragile]{Tamaño de letra}
    \small 
    El tamaño de letra para todo el documento puede ponerse como parámetro opcional de \texttt{documentclass}, p.e.
    \begin{latexsource}
\documentclass[8pt]{article}    
    \end{latexsource}
    El tamaño del texto puede cambiar localmente
    \begin{table} 
    \centering
    \begin{tabular}{l|l}
        \latexinline{\Huge Hola} & {\Huge Hola} \\
        \latexinline{\huge Hola} & {\huge Hola} \\
        \latexinline{\LARGE Hola} & {\LARGE Hola} \\
        \latexinline{\Large Hola} & {\Large Hola} \\
        \latexinline{\large Hola} & {\large Hola} \\
        \latexinline{\normalsize Hola} & {\normalsize Hola} \\
        \latexinline{\small Hola} & {\small Hola} \\
        \latexinline{\footnotesize Hola} & {\footnotesize Hola} \\
        \latexinline{\scriptsize Hola} & {\footnotesize Hola} \\
        \latexinline{\tiny Hola} & {\footnotesize Hola} \\
	\end{tabular}
    \end{table}
    Más sobre fuentes:

    \url{https://en.wikibooks.org/wiki/LaTeX/Fonts}
\end{frame}

\begin{frame}[fragile]{Fuente}
    El documentos \texttt{article} y \texttt{book} latex utiliza la fuente \texttt{Computer Modern}. Otras fuentes están disponibles
    \begin{table} 
        \centering
        \begin{tabular}{l|l}
            \latexinline{\usepackage{lmodern}} & \includegraphics{codigos/lmodern.pdf} \\
            \latexinline{\usepackage{mathptmx}} & \includegraphics{codigos/times.pdf} \\
            \latexinline{\usepackage{helvet}} & \includegraphics{codigos/helvetica.pdf} \\
        \end{tabular}
        \end{table}

    Añadir codificación de fuente tiene algunas ventajas. Es habitual añadir
    \begin{latexsource}
\usepackage[T1]{fontenc}
    \end{latexsource}

    Listas más extensas en 

    \url{https://www.overleaf.com/learn/latex/Font_typefaces}

    \url{https://tug.org/FontCatalogue/}\\

    Si utilizamos \texttt{XeLaTeX} o \texttt{LuaTeX} como compiladores, podemos usar el paquete \texttt{fontspec}
    
\end{frame}

\begin{frame}{Colores}
    El paquete \texttt{xcolor} nos permite utilizar diferentes colores, p.e.
    \begin{table} 
        \centering
        \begin{tabular}{l|l}
            \latexinline{\color{blue} Hola} & {\color{blue} Hola} \\
        \end{tabular}
        \end{table}
    Hay una extensa familia de espacios de nombres, y comando asociados. 

    \url{https://en.wikibooks.org/wiki/LaTeX/Colors}
\end{frame}

\begin{frame}[fragile]{Saltos de líneas}
    \tiny
    \latexinputandpdf{codigos/skip}    
\end{frame}
