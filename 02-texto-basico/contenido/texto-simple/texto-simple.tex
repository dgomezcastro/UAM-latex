% LTeX: language=es-ES

\section{Simplemente texto}


\begin{frame}[fragile]{Negrita, cursiva, ... }
    \begin{table} 
		\centering
	\begin{tabular}{l|l}
        \latexrowinlineandcompile{Hola} \\
        \latexrowinlineandcompile{\textbf{Hola}} \\
        \latexrowinlineandcompile{\textit{Hola}} \\
        \latexrowinlineandcompile{\textsl{Hola}} \\
        \latexrowinlineandcompile{\texttt{Hola}} \\
        \latexrowinlineandcompile{\textsc{Hola}} \\
        \latexrowinlineandcompile{\textrm{Hola}} \\
        \latexrowinlineandcompile{\underline{Hola}}
	\end{tabular}
	\end{table} 
    Otros paquetes incluyen más funcionalidades: \texttt{ulem}, \texttt{sout} permiten tachar texto, ... \\ 

    Con \latexinline{\emph} podemos resaltar una parte, independientemente del contexto\footnote{\latexinline{\emph} tiene un pequeño problema con \texttt{beamer}. Ver \href{https://tex.stackexchange.com/questions/89425/beamer-does-not-restore-roman-style-for-emph-inside-emph}{link}}
    \begin{table} 
		\centering
	\begin{tabular}{l|l}
        \latexrowinlineandcompile{Hola, \emph{Pablo.}}\\
        \latexrowinlineandcompile{\textit{Hola, \emph{Pablo.}}}
	\end{tabular}
	\end{table} 
\end{frame}

\begin{frame}[fragile]{Comillas}
    \LaTeX utiliza una convención curiosa abrir y cerrar comillas
    \begin{table}
    \centering
    \begin{tabular}{l|l}
        \latexrowinlineandcompile{Fue ``sin querer''}\\
        \mintinline{latex}{Fue "sin querer"} & Fue "sin querer"
	\end{tabular}
    \end{table}
\end{frame}

\begin{frame}[fragile]{Tamaño de letra}
    \small 
    El tamaño de letra para todo el documento puede ponerse como parámetro opcional de \texttt{documentclass}, p.e.
    \begin{latexsource}
\documentclass[8pt]{article}    
    \end{latexsource}
    El tamaño del texto puede cambiar localmente
    \begin{table} 
    \centering
    \begin{tabular}{l|l}
        \latexrowinlineandcompile{\Huge Hola}\\
        \latexrowinlineandcompile{\huge Hola} \\
        \latexrowinlineandcompile{\LARGE Hola} \\
        \latexrowinlineandcompile{\Large Hola}\\
        \latexrowinlineandcompile{\large Hola}\\
        \latexrowinlineandcompile{\normalsize Hola}\\
        \latexrowinlineandcompile{\small Hola} \\
        \latexrowinlineandcompile{\footnotesize Hola}\\
        \latexrowinlineandcompile{\scriptsize Hola} \\
        \latexrowinlineandcompile{\tiny Hola} 
	\end{tabular}
    \end{table}
    Más sobre fuentes:

    \url{https://en.wikibooks.org/wiki/LaTeX/Fonts}
\end{frame}

\begin{frame}[fragile]{Fuente}
    El documentos \texttt{article} y \texttt{book} latex utiliza la fuente \texttt{Computer Modern}. Otras fuentes están disponibles
    \begin{table} 
        \centering
        \begin{tabular}{l|l}
            \latexinline{\usepackage{lmodern}} & \includegraphics{codigos/lmodern.pdf} \\
            \latexinline{\usepackage{mathptmx}} & \includegraphics{codigos/times.pdf} \\
            \latexinline{\usepackage{helvet}} & \includegraphics{codigos/helvetica.pdf} \\
        \end{tabular}
        \end{table}
    Listas más extensas en 

    \url{https://www.overleaf.com/learn/latex/Font_typefaces}

    \url{https://tug.org/FontCatalogue/}\\

    Si utilizamos \texttt{XeLaTeX} o \texttt{LuaTeX} como compiladores, podemos usar el paquete \texttt{fontspec} \\ 


    Especificar la codificación de fuente tiene algunas ventajas. Es habitual includir
    \begin{latexsource}
\usepackage[T1]{fontenc}
    \end{latexsource}
\end{frame}

\begin{frame}{Colores}
    El paquete \texttt{xcolor} nos permite utilizar diferentes colores, p.e.
    \begin{table} 
        \centering
        \begin{tabular}{l|l}
            \latexinline{\color{blue} Hola} & {\color{blue} Hola} \\
        \end{tabular}
        \end{table}

    Lo habitual es querer colorear sólo algunas palabras. 
    
    Para ello podemos ponerlo encapsularlo:
    \latexinline{{\color{blue} Hola}} 

    o volver al color ``normal'': \latexinline{\color{blue} Hola \normalcolor}

    \bigskip 

    Hay una extensa familia de espacios de nombres, y comando asociados. 

    \url{https://en.wikibooks.org/wiki/LaTeX/Colors}
\end{frame}

\begin{frame}[fragile]{Saltos de líneas}
    \tiny
    \latexinputandpdf{codigos/skip}    
\end{frame}

\begin{frame}[fragile]{Listas (volveremos sobre esto en detalle)}

\latexinputandpdf{codigos/enumerate}

\end{frame} 

\begin{frame}[fragile]{Caracteres reservados y cómo escribirlos}
    \begin{table}
        \centering
        \begin{tabular}{llll}
            \latexinline{$} & modo matemático & \latexinline{\$, \textdollar} & \$ \\
            \latexinline \\
            \latexinline{&} & tabulación &
            \latexrowinlineandcompile{\&} \\
            \latexinline{^, _} & super y sub índices & 
            \latexrowinlineandcompile{\^{}, \_} \\
            \latexinline{\ } & comienzo de comando &
            \latexrowinlineandcompile{\textbackslash}\\
            \latexinline{{}} & inicio y fin de bloque & 
            \latexrowinlineandcompile{\{ \}}\\ 
        \end{tabular}
    \end{table}
\end{frame}
