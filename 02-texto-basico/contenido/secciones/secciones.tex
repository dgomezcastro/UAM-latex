\section{Secciones e índice}
\frame{\sectionpage}

\begin{frame}[fragile]
	\frametitle{Secciones e índices} 
	
	Naturalmente \LaTeX\ permite estructurar el documento en secciones. \\ 
	
	\pause 
	
	Para empezar una nueva sección con título basta con indicarlo y dar un título utilizando los siguientes comandos:
	
	\begin{table} 
		\centering
	\begin{tabular}{c|l}
	-1 & \latexinline{\part{titulo}} \\
	0 &	\latexinline{\chapter{titulo}} \\
		1	& \latexinline{\section{titulo}} \\
		2	& \latexinline{\subsection{titulo}} \\
		3	& \latexinline{\subsubsection{titulo}} \\
		4	& \latexinline{\paragraph{titulo}} \\
		5	& \latexinline{\subparagraph{titulo}}
	\end{tabular}
	\end{table} 

	\pause 

	Se puede generar el índice introduciendo \latexinline{\tableofcontents}. Este comando admite parámetros optativos.  \\ 
	
	\pause 
	
	También se pueden hacer otras tablas de contenidos: \latexinline{\listoffigures}, \latexinline {\listoftables}.
\end{frame}