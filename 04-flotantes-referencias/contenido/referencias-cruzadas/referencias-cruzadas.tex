
\section{Etiquetas} 
\begin{frame}[fragile]
\frametitle{Referencias}
Para establecer una etiqueta a la que poder llamar se emplea
\begin{latexsource}
\label{<label>}
\end{latexsource}
Es habitual emplear etiquetas de la forma
\begin{latexsource}
<label>=thm:euclides, fig:gauss, eq:divergencia
\end{latexsource}
No se pueden emplear acentos. 
\end{frame}

\begin{frame}[fragile]
\frametitle{Referencias}
\framesubtitle{Ejemplo}
\begin{columns}
\column{.4\textwidth}
\begin{block}{Código}
	\scriptsize
	\begin{latexsource}
		\begin{<thm>} \label{<label>}

		\end{<thm>}    
	\end{latexsource}
\end{block}
\column{.4\textwidth}
\begin{block}{Código}
	\scriptsize
	\begin{latexsource}
		\begin{figure}

		\includegraphics{<path>}
		\caption{<caption>}
		\label{<label>}

		\end{figure}   
	\end{latexsource}
\end{block}
\end{columns}
\end{frame}

\section{Referencias básicas de \LaTeX}

\begin{frame}[fragile]
\frametitle{Referencias: El comando \latexinline{\ref}}
	\small
	Podemos llamar al número de una etiqueta mediante  \latexinline{\ref{<label>}}. \\

\begin{columns}
\column{.45\textwidth}
\visible<2>{
\begin{block}{Solución}
	\tiny
	\latexfile{codigos/ref2.tex}
\end{block}}
\column{.5\textwidth}
	\textbf{Ejercicio:} Escribir el código correspondiente a la siguiente salida: 
\fbox{
\includegraphics[trim=4cm 15cm 4cm 3cm,clip,   scale=.4]{codigos/ref2.pdf}
}
\end{columns}
\end{frame}

\begin{frame}
\frametitle{Referencias avanzadas: el paquete \latexinline{\cleveref}}
	Introduciendo este paquete tenemos acceso al comando \latexinline{\Cref}. \\ 
	
	Esta función actúa como \latexinline{\ref}, pero incluye automáticamente el tipo de objeto referenciado: Teorema, Figura... \\[-2ex]
	\pause 
	\begin{columns}
		\column{.45\textwidth}
			\begin{block}{Código}
				\tiny
				\latexfile{codigos/ref3.tex}
		\end{block}
		\column{.5\textwidth}
		\fbox{
			\includegraphics[trim=4cm 15cm 4cm 3cm,clip,   scale=.4]{codigos/ref3.pdf}
			}
	\end{columns}
	
	
\end{frame}