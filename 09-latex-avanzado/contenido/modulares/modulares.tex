\section<{Ficheros modulares}
\begin{frame}[fragile]
	\frametitle{Ficheros modulares: \texttt{input}}
	Escribir un libro completo en un único archivo no es cómodo. Por eso \LaTeX\ permite escribir modularmente. 
	
	Podemos escribir en diferentes archivos \texttt{.tex}, y luego juntarlos en un principal.
	\begin{columns}
		\column{.5\textwidth}
		\begin{block}{Código [\texttt{modular.tex}]}
			\latexfile[highlightlines={3-4}]{codigos/modular.tex}
		\end{block}
		\begin{block}{Código [\texttt{modulo1.tex}]}
			\latexfile{codigos/modulo1.tex}
		\end{block}
		\begin{block}{Código [\texttt{modulo2.tex}]}
			\latexfile{codigos/modulo2.tex}
		\end{block}
		\column{.4\textwidth}
		\begin{figure}
			\centering 
			\fbox{\includegraphics[]{codigos/modular.pdf}}
			\caption{Resultado de compilar \texttt{modular.tex}}
		\end{figure} 
	\end{columns}
\end{frame}

\begin{frame}[fragile]{El paquete \texttt{import}}
	% TODO
\end{frame}