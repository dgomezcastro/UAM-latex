% LTeX: language=es-ES


\section{Distintos tipos de fórmulas}



\begin{frame}[fragile]
    \frametitle{Escribiendo fórmulas}
    Hay diferentes entornos para escribir fórmulas:
    \begin{columns}
    \column{0.4\textwidth}
    \begin{enumerate}
        \item<1-> \alert<1>{En línea \\(\latexinline{\(...\)} o \latexinline{$...$})}
        \item<2-> \alert<2>{Presentada \\(\latexinline{\[...\]} o \latexinline{$$...$$})}
        \item<3-> \alert<3>{\latexinline{equation} }
        \item<4-> \alert<4>{\latexinline{align}}
    \end{enumerate}
    \column{0.5\textwidth}
        \only<1>{
            \begin{block}{Código}
                \small
                 \latexinline{Puedo escribir $e^{i\pi} + 1 = 0$}
            \end{block}
%            \vspace{.5cm}
            \fbox{
            \begin{minipage}{\textwidth} 
                Puedo escribir $e^{i\pi}+1 = 0$
            \end{minipage}  }
        }
        \only<2>{
            \begin{block}{Código}
                \latexinline{Puedo escribir} \\
                \latexinline{\[ e^{i\pi} + 1 = 0 \]}
            \end{block}
            %            \vspace{.5cm}
            \fbox{
                \begin{minipage}{\textwidth} 
                    Puedo escribir $$e^{i\pi}+1 = 0$$
                \end{minipage}  }
            }
        \only<3>{
            \begin{block}{Código}
            \latexfile{codigos/equation.tex}
            \end{block}
            %            \vspace{.5cm}
            \fbox{
                \begin{minipage}{\textwidth} 
                 Puedo escribir
\begin{equation}
    e^{i\pi} + 1 = 0 
\end{equation}
                \end{minipage}  }
            }
        \only<4>{
            \begin{block}{Código}
                \latexfile{codigos/align.tex}
            \end{block}
            %            \vspace{.5cm}
            \fbox{
                \begin{minipage}{\textwidth} 
                    Puedo escribir
\begin{align}
    e^{i\pi} + 1 &= 0  \\
    e^{i\pi} &= -1
\end{align}
                \end{minipage}  }
            }
    \end{columns}

    \pause 

    Añadir \latexinline{*} al final de estos comandos (\latexinline{equation*,align*},...) elimina la numeración.
\end{frame}

\begin{frame}[fragile]
    \frametitle{La ecuación más bella del mundo}
    La ecuación de Euler, popular por contener algunas de las más importantes constantes matemáticas puede escribirse
    % \latexinlineandcompile{$ e^{i\pi} + 1 = 0 $}
    \latexinputandcompile{codigos/euler.tex}
\end{frame}

\section{Símbolos}


\begin{frame}[fragile]
\frametitle{Símbolos útiles}
\begin{tabular}{cl  cl cl}
	\latexrowcompileandinline{$+$} & 
    \latexrowcompileandinline{$\varepsilon$} & 
    \latexrowcompileandinline{$\frac a b$} \\
	\latexrowcompileandinline{$-$} & 
    \latexrowcompileandinline{$\delta$}& 
    \latexrowcompileandinline{$a^b$}  \\
	\latexrowcompileandinline{$\times$} &
    \latexrowcompileandinline{$\partial$} &
    \latexrowcompileandinline{$\sqrt a$} \\
	\latexrowcompileandinline{$\div$} & 
    \latexrowcompileandinline{$\Omega$} & 
    \latexrowcompileandinline{$\max, \min$} \\
	\latexrowcompileandinline{$\cdot$}&
    \latexrowcompileandinline{$\pi$} &
    \latexrowcompileandinline{$\cos, \sin$}\\
	\latexrowcompileandinline{$\oplus$} & 
	\latexrowcompileandinline{$\otimes$} &
    \latexrowcompileandinline{$\exp, \log$}
\end{tabular}

\latexinputandcompile{codigos/sum.tex}

\vspace{1ex}

La web \href{http://detexify.kirelabs.org/classify.html}{Detexify} permite buscar símbolos.
\end{frame}

\begin{frame}{Subíndices y superíndices}
	Los subíndices de sumarios e integrales cambian de formato presentado a en línea
    \latexinputandcompile{codigos/ejemplo-subindices.tex}
\end{frame} 

\begin{frame}[fragile]{Subíndices y superíndices}
	Además la función \latexinline{\substack} es muy útil
	\latexinputandcompile{codigos/ejemplo-subindices-2.tex}
\end{frame}

\begin{frame}[fragile]{Texto dentro de las fórmulas}
    \latexinputandcompile{codigos/texto.tex}
\end{frame}

\begin{frame}[fragile]{Espaciado dentro de fórmulas}
    \latexinputandcompile{codigos/espaciado.tex}
\end{frame}

\begin{frame}[fragile]{Fuentes matemáticas}
    En el formato matemático hay diferentes fuentes
    \begin{table} 
    \centering
    \begin{tabular}{cl}
        \latexrowcompileandinline{$a$} \\
        \latexrowcompileandinline{$\mathrm{a}$} \\
        \latexrowcompileandinline{$\mathbf{a}$} \\
        \latexrowcompileandinline{$\mathsf{a}$} \\
        \latexrowcompileandinline{$\vec{a}$}
    \end{tabular}
    \end{table}
    El paquete \texttt{amssymb} añade
    \begin{table} 
        \centering
        \begin{tabular}{cl}
        \latexrowcompileandinline{$\mathbb{R}$} \\
        \latexrowcompileandinline{$\mathcal{R}$} \\
        \latexrowcompileandinline{$\mathfrak{R}$} \\
    \end{tabular}
    \end{table}
\end{frame}

\begin{frame}[fragile]{Paréntesis y delimitadores}
    \begin{table} 
        \centering
        \begin{tabular}{cl}
        \latexrowinlineandcompile{$(x+y)z$} \\
        \latexrowinlineandcompile{$[x+y]z$} \\
        \latexrowinlineandcompile{$\{x+y \mid x,y \ge 0\}$} \\
        \latexrowinlineandcompile{$\langle \rangle$} \\
        \latexrowinlineandcompile{$\lceil, \rceil, \lfloor, \rfloor$} \\
        \latexrowinlineandcompile{$\big( \Big( \bigg( \Bigg($} \\
        \latexrowinlineandcompile{$\left(x^{2^2} + y^{2^2}\right)$} \\
    \end{tabular}
    \end{table}
\end{frame}

% \begin{frame}[fragile]
%     Es habitual crear comandos para este tipo de letras
%     \latexinputandpdf{codigos/R}
% \end{frame}

\section{Matrices y diagramas}


\begin{frame}[fragile]{Matrices}
    Las matrices se introducen siempre en entornos matemáticos. Maple y matlab permiten exportar matrices a \LaTeX. 
    Hay distintos tipos de matrices predeterminadas en el paquete \texttt{amsmath}.
    \begin{enumerate}
        \item \texttt{matrix} Sin bordes
        \item \texttt{pmatrix} Entre ()
        \item \texttt{vmatrix} Entre $| \ |$
        \item \texttt{bmatrix} Entre [ ]
    \end{enumerate}
\end{frame}


\begin{frame}[fragile]
    \frametitle{Matrices}
    \framesubtitle{Ejemplo}
    \latexinputandcompile{codigos/matrix1.tex}
\end{frame}

\begin{frame}[fragile]
    \frametitle{Matrices}
    \framesubtitle{Ejemplo}
    
    \latexinputandcompile{codigos/matrix2.tex}
\end{frame}

\begin{frame}[fragile]{Diagramas}
    \textbf{El paquete \texttt{tikzcd}} Que depende de la librería \texttt{tikz}. Hablaremos de ella más adelante 

    \textbf{El paquete \texttt{xy-pic}}
    Este paquete se emplea para hacer todo tipo de gráficos, por ejemplo el diagrama
    \[ 
        \xymatrix{
        A \ar[r]^f \ar[dr]_{g\circ f} & B \ar[d]^g \\
        & C 
        }
    \] 
    Tiene infinidad de opciones.
    Será el que cubramos en esta sesión.
\end{frame}

\begin{frame}[fragile]{\texttt{xymatrix}} 
    Es la manera más sencilla de introducir diagramas. Los elementos que se conectarán por flechas se introducen en las posiciones de una matriz, de tipo \texttt{xymatrix}

    \latexinputandcompile{codigos/xy-empty.tex}

    Se puede introducir una \texttt{xymatrix} dentro o fuera de fórmulas, pero deberemos tener cuidado con el contenido.
\end{frame}

\begin{frame}[fragile]{Las flechas}
    Dentro de una \texttt{xymatrix} podemos introducir flechas con el comando \latexinline{\ar}

    Admite varios modificadores
    \begin{columns}
    \column{0.4\textwidth}
    \begin{enumerate}
    \item <1-> \alert<1>{Destino} \only<1>{Colocando la flecha en la casilla de la que parte se coloca un cadena de cuantas casillas a derecha o izquierda y arriba o abajo está el destino.

    \latexinline{\ar[<hop>]}}
    \item <2-> \alert<2>{Etiqueta} \only<2>{Se puede escribir sobre las letras}
    \item <3-> \alert<3>{Tipo} \only<3>{Hay distintos tipos de base, cuerpos y cabezas de flecha

    \latexinline{\ar@{<type>}[<hop>]}}
    \item<4-> \alert<4>{Curvatura} \only<4>{
    Podemos curvar las flechas hacia arriba y hacia abajo, para evitar que se corten, o solo para quede más estiloso
    \latexinline{\ar @/<curve>/ [<hop>]}
    }
    \item<5-> \alert<5>{Entrada y salida} \only<5>{Si queremos que la flecha salga desde una parte en concreto de la celda podemos especificarlo

    \latexinline{\ar@(<in>,<out>)[<hop>]}}
    \end{enumerate}

    \column{0.4\textwidth}
    \fbox{
    \begin{minipage}{\textwidth}
    \centering
    \begin{tabular}{l l} %
    \only<1>{%
    \texttt{u} & arriba\\
    \texttt{d} & abajo \\
    \texttt{r} & derecha\\
    \texttt{l} & izquierda\\
    \texttt{ } & a si misma}%
    \only<2>{%
    \latexinline{\ar[r]^{f}} & \xymatrix{ a \ar[r]^f & b } \\
    \latexinline{\ar[r]_{f}} & \xymatrix{ a \ar[r]_f & b } \\
    \latexinline{\ar[r]|{f}} & \xymatrix{ a \ar[r]|f & b }
    }%
    \only<3>{%
    \latexinline{@{=>}} & \xymatrix{ a \ar@{=>}[r] & b } \\
    \latexinline{@{.>}} & \xymatrix{ a \ar@{.>}[r] & b } \\
    \latexinline{@{:>}} & \xymatrix{ a \ar@{:>}[r] & b } \\
    \latexinline{@{~>}} & \xymatrix{ a \ar@{~>}[r] & b } \\
    \latexinline{@{-->}} & \xymatrix{ a \ar@{-->}[r] & b} \\
    \latexinline{@{|->}} & \xymatrix{ a \ar@{|->}[r] & b} 
    }%
    \only<4>{%
    \latexinline{@/_/} & \xymatrix{ a \ar@/_/[r] & b } \\
    \latexinline{@/^/} & \xymatrix{ a \ar@/^/[r] & b } \\
    \latexinline{@/_1mm/} & \xymatrix{ a \ar@/_1mm/[r] & b }
    }%
    \only<5>{%
    \latexinline{@(u,d)[r]} & \xymatrix{ a \ar@(u,d)[r] & b } \\
    \latexinline{@(ur,dr)[]} &\xymatrix{a \ar@(ur,dr)[] & b}
    }
    \end{tabular}
    \end{minipage}    
    }
    \end{columns}

\end{frame}

\begin{frame}[fragile]
\frametitle{Las flechas}
\framesubtitle{Ejercicio}

Escriba el siguiente diagrama:
\scriptsize
\[ 
\xymatrix{
    A \ar@/_2ex/[ddr] \ar[dr]|f \ar@/^2ex/[drr] \\
      & B \ar@{-->}[r] \ar@{^(->}[d] & C\ar@(dr,ur)[]_{id} \\
      & D
    }
\]

\uncover<2>{\begin{block}{Código (\texttt{ejercicio3.tex})}
\scriptsize
\latexfile{codigos/xy.tex}
\end{block}}
\end{frame}

\begin{frame}[fragile]
\frametitle{El paquete \texttt{xy-pic} y el paquete \texttt{babel}}
El paquete \texttt{babel} entra en conflicto con \latexinline{@} así que si queremos hacer buenos diagramas debemos desactivarlo. Empleando \latexinline{inputenc} con \texttt{utf8} no tendremos problemas con los acentos. Debemos cambiar los nombres de capítulos y secciones.
Para ello
\latexinline{\renewcommand{<command>}{<new_name>}}\\
\vspace{.5cm}

\centering
\scriptsize
\begin{tabular}{l|l}
\latexinline{\abstractname}&	 Abstract \\
\latexinline{\appendixname} &	 Appendix\\
\latexinline{\bibname}&	 Bibliography (report,book)\\
\latexinline{\chaptername} &	 Chapter (report,book)\\
\latexinline{\contentsname} &	 Contents\\
\latexinline{\figurename}	& Figure (for captions)\\
\latexinline{\indexname}	& Index\\
\latexinline {\listfigurename}&	 List of Figures\\
\latexinline{\listtablename}&	 List of Tables\\
\latexinline{\tablename}&	 Table (for caption)\\
\end{tabular}

\end{frame}