% LTeX: language=es-ES

\section{Creando teoremas}
\begin{frame}[fragile]
    \frametitle{Teoremas}
    \framesubtitle{El paquete \texttt{amsthm}}
%Existen más paquetes como \texttt{ntheorem}.
A la hora definir un teorema debemos tener en cuenta tres cosas
\begin{columns}
\column{0.5\textwidth}
\begin{enumerate}
	\item <1-> \alert<1>{El estilo}\only<1>{: Los teoremas se escriben en cursiva, mientras que las definiciones se escriben con fuente normal.}
	\item <2-> \alert<2>{El nombre}\only<2>{: Debemos poner un nombre de entorno \latexinline{<env>}, ya sea teorema (por ejemplo \latexinline{<env>=teorema}) un nombre para mostrar en el documento (por ejemplo \latexinline{<name>=Teorema})}
	\item <3-> \alert<3>{La numeración}\only<3>{: Podemos numerar los teoremas de diferentes maneras}
	\only<4->{\begin{enumerate}[a)]
		\item <4-> Con su propio contador: \only<4>{El contador se crea por defecto si no decimos nada más, y se nombra automáticamente como \latexinline{<env>}}
		\item <5-> Siguiendo la numeración de otro teorema ya definido
		\item <6->Supeditada a otro contador, por ejemplo la sección. \only<6>{En este caso el contador de tipo a) lleva como predecesor el otro contador, y se resetea al cambiar el contador al que supedita}
		\end{enumerate}}
\end{enumerate}
	\column{0.5\textwidth}
\begin{block}{Código}
\scriptsize
\latexinline{\documentclass}\\

(...)\\

\alert<1>{ \latexinline{\theoremstyle{<style>}} }\\

\latexinline{\newtheorem}\alert<2>{\latexinline}\only<5>{\alert{\texttt{[<counter>]}}}\alert<2>{\latexinline}\only<6>{\alert{\texttt{[<counter>]}}}\\

(...)\\

\latexinline{\begin{document}}\\

\end{block}
\end{columns}
\only<1>{
Hay tres estilos predefinidos:\\

\hspace{1cm}~
		\begin{tabular}{r|l}
		\texttt{plain} & \textbf{Theorem 1.}\textit{Theorem text.} \\
		\texttt{definition} & \textbf{Definition 1.} {Definition text.} \\
		\texttt{remark} & \textit{Remark 1.}{Remark text.} \\
		\end{tabular}}
\end{frame}


\begin{frame}[fragile]
    \frametitle{Teoremas}
    \framesubtitle{Ejemplo}
Este es el aspecto de un teorema normal definido con el paquete \texttt{amsthm}.
\begin{columns}
	\column{0.5\textwidth}
	\begin{block}{Código (\texttt{thm1.tex})}
		\scriptsize
		\latexfile{codigos/thm1.tex}
	\end{block}
	\column{0.4\textwidth}
	\fbox{\begin{minipage}{\textwidth}
		\textbf{Teorema 1} (Euclides)~\textbf{.} \textit{No existe un primo mayor que el resto.}
		\end{minipage}
	}
\end{columns}
\end{frame}