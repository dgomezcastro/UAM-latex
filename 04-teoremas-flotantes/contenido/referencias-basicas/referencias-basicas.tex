% LTeX: language=es-ES

\section{Referencias básicas de \LaTeX}

\begin{frame}[fragile]
\frametitle{Etiquetas}
Algunos entornos están numerados y se pueden etiquetar con
\begin{latexsource}
\label{<label>}
\end{latexsource}
Es habitual emplear etiquetas de la forma
\begin{latexsource}
<label>=thm:euclides, fig:gauss, eq:divergencia
\end{latexsource}
No se pueden emplear acentos.

Se citan con \latexinline{\ref} (o \latexinline{\eqref} de \texttt{amsmath})

\latexinputandcompile{codigos/label.tex}

\end{frame}

\begin{frame}[fragile]
\frametitle{Referencias}
\framesubtitle{Ejemplo}
\begin{columns}
\column{.4\textwidth}
\begin{block}{Código}
	\scriptsize
	\begin{latexsource}
		\begin{<thm>} \label{<label>}

		\end{<thm>}    
	\end{latexsource}
\end{block}
\column{.4\textwidth}
\begin{block}{Código}
	\scriptsize
	\begin{latexsource}
		\begin{figure}

		\includegraphics{<path>}
		\caption{<caption>}
		\label{<label>}

		\end{figure}   
	\end{latexsource}
\end{block}
\end{columns}
\end{frame}


\begin{frame}[fragile]
\frametitle{Referencias: El comando \latexinline{\ref}}
	\small
	Podemos llamar al número de una etiqueta mediante  \latexinline{\ref{<label>}}. \\

\begin{columns}
\column{.45\textwidth}
	% \visible<2>
	{
	\begin{block}{Ejemplo}
		\tiny
		\latexfile{codigos/ref2.tex}
	\end{block}}
	\column{.5\textwidth}
		\fbox{
		\includegraphics[trim=4cm 15cm 4cm 3cm,clip,   scale=.4]{codigos/ref2.pdf}
	}
\end{columns}
\end{frame}

\begin{frame}[fragile]{Etiquetas especiales para ecuaciones con \latexinline{\tag}}
	\latexinputandcompile{codigos/tag.tex}
\end{frame}

\begin{frame}{Más sobre referencias}
	Más adelante haremos una sesión sobre referencias cruzadas.\\ 

	Si queremos que nuestras referencias tengan hipervínculos, utilizaremos el paquete \texttt{hyperref} del que hablaremos más adelante.\\

	El paquete \texttt{hyperref} incluye algunos comando útiles como \latexinline{\url} y \latexinline{\href}.
\end{frame}