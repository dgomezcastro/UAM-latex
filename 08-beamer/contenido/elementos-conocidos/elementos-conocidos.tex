\section{Usando los elementos conocidos}


\begin{frame}[fragile]{Título y autor}
    \small
    \latexinputandpdf{codigos/beamer-maketitle}

    \bigskip

    \latexinline{\title, \author, \date} admiten un parámetro opcional. 

    Este parámetro configura el título a mostrar en diferentes lugares \\(en particular el pie de página)
\end{frame}

\begin{frame}[fragile]{Teoremas}
    \footnotesize
    \latexinputandpdf{codigos/beamer-thm}
\end{frame}

\begin{frame}{Secciones}
    \tiny
    \latexinputandpdf{codigos/section}

    Si queremos que \latexinline{\sectionpage} (y otros elementos) funcionen bien en castellano, debemos pasar la opción \texttt{spanish} al \latexinline{\documentclass}:

    \latexinline{\documentclass[spanish]{beamer}}
\end{frame}

\begin{frame}[fragile]{Figuras}
    \footnotesize
    \latexinputandpdf{codigos/beamer-figure}
\end{frame}

\begin{frame}{Bibliografía}
    \small
    \begin{columns}
        \column{0.55\textwidth}
        \begin{block}{Código}
            \latexfile{codigos/ejemplo-biblatex.tex}
        \end{block}
        \column{0.35\textwidth}
        \centering
        \begin{minipage}{\textwidth}
            \fbox{\includegraphics[width=\textwidth,page=1]{codigos/ejemplo-biblatex.pdf}}
            \fbox{\includegraphics[width=\textwidth,page=2]{codigos/ejemplo-biblatex.pdf}}        \end{minipage}
      \end{columns}
\end{frame}

\begin{frame}{Elementos frágiles}

    Algunos elementos como \latexinline{\url} o incluir códigos con \texttt{verbatim} o \texttt{minted} tiene problemas con \texttt{beamer}. 

    \bigskip 

    Para evitar estos problemas podemos marcar la diapositiva como \texttt{fragile} con
    \begin{center}
        \latexinline{\begin{frame}[fragile]{<título>}}
    \end{center}
    
    \bigskip

    Hablaremos de esto más adelante.
\end{frame}