% LTeX: language=es-ES

\section{Dándole formato a una diapositiva}

\begin{frame}[fragile]{\latexinline{\alert}}

    \begin{quote}
        beamer offers the command
        \latexinline{\alert}, which is used like \latexinline{\emph} and, by default, typesets its argument in bright red.
    \end{quote}

\end{frame}

\begin{frame}[fragile]{Columnas}
    \small
    \latexinputandpdf{codigos/columns}
\end{frame}

\begin{frame}[fragile]{Pausas}
    \small
    \begin{columns}
        \column{0.55\textwidth}
        \begin{block}{Código}
            \latexfile{codigos/pauses.tex}
        \end{block}
        \column{0.25\textwidth}
        \centering
        \begin{minipage}{\textwidth}
            \fbox{\includegraphics[width=\textwidth,page=1]{codigos/pauses.pdf}}
            \fbox{\includegraphics[width=\textwidth,page=2]{codigos/pauses.pdf}}        
            \fbox{\includegraphics[width=\textwidth,page=3]{codigos/pauses.pdf}}        
        \end{minipage}
      \end{columns}
\end{frame}

\begin{frame}[fragile]{Overlays}
    \latexinline{\pause} Crea elementos llamados \texttt{overlay}, y los ordena consecutivamente. 

    Se puede controlar de manera sencilla cuando aparecen algunos elementos usando \texttt{specifications}
    \begin{itemize}
        \item \latexinline{\only<>{}} 
        reveal content , does NOT
        occupy space otherwise
        \item \latexinline{\uncover<>{}} 
        reveal content, DOES
        occupy space otherwise
        \item \latexinline{\visible<>{}}
        \item \latexinline{\invisible<>{}}
        \item \latexinline{\item<>} 
     
        \item \latexinline{\textbf<>{}} 
        \item \latexinline{\textit<>{}} 
        \item \latexinline{\color<>[]{}} 
        controls when to
        change color of
        text
        \item \latexinline{\alt<>{}{}} 
        reveals first argument when specification is true, otherwise reveals second argument
        \item \latexinline{\alert<>{}} 
        controls when to highlight text (default red)
    \end{itemize}
\end{frame}

\begin{frame}[fragile]{Overlays}
    Se puede especificar
    \begin{itemize}
        \item Una diapositiva concreta donde se visible: 
        
        \latexinline{<1>}
        
        \item Rangos de diapositivas a partir de la cual sea visible:
        
        \latexinline{<1->}, \latexinline{<1-3>}, \latexinline{<-3>}

        \item overlay incremental 
        
        \latexinline{<+->}
    \end{itemize}

    \bigskip

    \pause 

    \begin{itemize}
        \item \latexinline{<3-| alert@3>}
        \begin{quote}
            Some overlay specification-aware commands cannot handle not only normal overlay specifications, but
            also so called action specifications. In an action specification, the list of slide numbers and ranges is prefixed
            by ⟨action⟩@, where ⟨action⟩ is the name of a certain action to be taken on the specified slides
        \end{quote}
    
    \end{itemize}

    Más detalles en la documentación de beamer (\href{https://tug.ctan.org/macros/latex/contrib/beamer/doc/beameruserguide.pdf#subsection.9.6}{link}).
\end{frame}

\begin{frame}[fragile]
    \tiny
    \begin{columns}
        \column{0.7\textwidth}
        \begin{block}{Código}
            \latexfile{codigos/overlay.tex}
        \end{block}
        \column{0.2\textwidth}
        \centering
        \begin{minipage}{\textwidth}
            \fbox{\includegraphics[width=\textwidth,page=1]{codigos/overlay.pdf}}
            \fbox{\includegraphics[width=\textwidth,page=2]{codigos/overlay.pdf}}        
            \fbox{\includegraphics[width=\textwidth,page=3]{codigos/overlay.pdf}}        
            \fbox{\includegraphics[width=\textwidth,page=4]{codigos/overlay.pdf}}        
        \end{minipage}
      \end{columns}
\end{frame}


\begin{frame}{\texttt{allowframebreaks}}
    Cuando una diapositiva es demasiada larga podemos partir en varias con la opción \texttt{allowframebreaks}.

    \bigskip 

    La división ocurre de manera automática\footnote{igual que los saltos de página en \texttt{article}}, pero podemos forzarlos con \latexinline{\framebreak}.

    \bigskip

    Es recomendable hacerlo con la bibliografía.
\end{frame}