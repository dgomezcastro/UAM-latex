% LTeX: language=es-ES

\section{\texttt{beamer} avanzado}

\begin{frame}{\texttt{block}}
    Los teoremas que hemos visto antes aprovechan un entorno de \texttt{beamer} llamado \texttt{block}.

    \latexinputandpdf{codigos/block}

\end{frame}

\begin{frame}{\texttt{block} y temas}

    \latexinputandpdf{codigos/block-Madrid}

    \bigskip

    También está disponible \texttt{alertblock} y \texttt{example}.
\end{frame}

\begin{frame}[fragile]{Opciones}
    En la cabecera se puede fijar
    
    \latexinline{\setbeamertemplate{blocks}[rounded][shadow=true]}

    \bigskip 

    También se puede fijar los colores de los distintos tipos de bloques
    \begin{latexsource}
\setbeamercolor{block title}{bg=cyan, fg=white}
\setbeamercolor{block body}{bg=cyan!10}
    \end{latexsource}
\end{frame}


\begin{frame}[fragile]{Hipervínculos y botones}
    \footnotesize
    \begin{columns}
        \column{0.55\textwidth}
        \begin{block}{Código}
            \href{https://tex.stackexchange.com/questions/424650/how-to-jump-to-a-slide-in-a-beamer-presentation}{StackOverflow}
            \latexfile[firstline=2]{codigos/goto.tex}
        \end{block}
        \column{0.25\textwidth}
        \centering
        \begin{minipage}{\textwidth}
            \fbox{\includegraphics[width=\textwidth,page=1]{codigos/goto.pdf}}
            \fbox{\includegraphics[width=\textwidth,page=2]{codigos/goto.pdf}}        
            \fbox{\includegraphics[width=\textwidth,page=3]{codigos/goto.pdf}}        
        \end{minipage}
      \end{columns}
      \bigskip
\end{frame}

\begin{frame}{Formatos opcionales}
    \latexinline{\documentclass[]{beamer}} admite parámetros opcionales:
    
    \begin{itemize}
        \item \texttt{handout}: sin overlays (pauses).
        \item \texttt{draft}: sin imágenes (acelera la compilación)
    \end{itemize}
    
\end{frame}

\begin{frame}[fragile]{\latexinline{\mode}}
    \latexinline{\documentclass{article}} es capaz de procesar entornos \texttt{frame}. 

    \bigskip 

    Así, podemos crear unas notas a partir de un documento \texttt{beamer}. Si queremos hay partes del código (especialmente la cabecera) que sólo se procesen en \texttt{beamer} o en \texttt{article} podemos incluirlas en un 

    \latexinline{\mode<presentation>} 
    
    \latexinline{\mode<draft>}
    
    \latexinline{\mode<article>}
\end{frame}


\begin{frame}[fragile]{Póster de congreso}
    Podemos utilizar \texttt{beamer} para crear poster de congresos. 
    Más detalles en \href{https://www.overleaf.com/learn/latex/Posters}{link}.

    \bigskip

    La alternativa es utilizar \texttt{tikzposter}.
\end{frame}