\section{Primeros pasos}

\begin{frame}[fragile]
    \frametitle{El paquete \texttt{beamer}}
    El estilo \texttt{beamer} nos permite hacer presentaciones en \LaTeX \ como esta.
\end{frame}

\begin{frame}{Ejemplo mínimo}
    \latexinputandpdf{codigos/mwe}
\end{frame}

\begin{frame}{navigation bar}
    Primero,\footnote{por decencia} quitamos la barra de navegación añadiendo a la cabecera
    
    \begin{center} 
        \latexinline{\beamertemplatenavigationsymbolsempty}
    \end{center}
\end{frame}

%\subsection{La cabecera}
\begin{frame}[fragile]
    {Temas}
    
    \small

    \latexinputandpdf{codigos/beamer1}
	
	\vspace{3ex}

	Para ver diferentes estilos y colores visitar \url{https://hartwork.org/beamer-theme-matrix/}
\end{frame}

\begin{frame}[fragile]{Diapositivas: entorno \texttt{frame}}
    \begin{block}{Código}
        \latexfile{codigos/frame.tex}
    \end{block}
    Volveremos sobre \texttt{overlay} más adelante.
\end{frame}