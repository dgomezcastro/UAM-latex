\section{Descomponer el documento en ficheros}
\begin{frame}[fragile]{\texttt{input}}
	\small
	Escribir un libro completo en un único archivo no es cómodo. 
	
	Por eso \LaTeX\ permite escribir modularmente. 
	
	Podemos escribir en diferentes archivos \texttt{.tex}, y luego juntarlos en un principal.
	\begin{columns}
		\column{.5\textwidth}
		\begin{block}{Código [\texttt{modular.tex}]}
			\latexfile[highlightlines={3-4}]{codigos/modular.tex}
		\end{block}
		\begin{block}{Código [\texttt{modulo1.tex}]}
			\latexfile{codigos/modulo1.tex}
		\end{block}
		\begin{block}{Código [\texttt{modulo2.tex}]}
			\latexfile{codigos/modulo2.tex}
		\end{block}
		\column{.4\textwidth}
		\begin{figure}
			\centering 
			\fbox{\includegraphics[]{codigos/modular.pdf}}
			\caption{Resultado de compilar \texttt{modular.tex}}
		\end{figure} 
	\end{columns}
\end{frame}

\begin{frame}[fragile]{\texttt{include}}
	\url{https://tex.stackexchange.com/questions/246/when-should-i-use-input-vs-include} (1405 upvotes)

	\bigskip 

	\latexinline{\input{filename}} imports the commands from \texttt{filename.tex} into the target file; it's equivalent to typing all the commands from \texttt{filename.tex} right into the current file where the \latexinline{\input} line is.

	\bigskip 

	\latexinline{\include{filename}} essentially does a \latexinline{\clearpage} before and after \latexinline{\input{filename}}, together with some magic to switch to another \texttt{.aux} file, and omits the inclusion at all if you have an \latexinline{\includeonly} without the filename in the argument. This is primarily useful when you have a big project on a slow computer; changing one of the include targets won't force you to regenerate the outputs of all the rest.
	
	\latexinline{\include{filename}} gets you the speed bonus, but it also can't be nested, can't appear in the preamble, and forces page breaks around the included text.
\end{frame} 

\begin{frame}[fragile]{\latexinline{\input} anidados}
	\latexinline{\input} translada copia-y-pega el archivo indicado.
	
	Los \texttt{path} deben darse siempre con respecto al archivo que compila.

	\begin{columns}
	\column{0.45\textwidth}
		\begin{itemize}
			\item \texttt{main.tex}
			\item folder1
			\begin{itemize}
				\item \texttt{file1.tex}
				\item folder2
				\begin{itemize}
					\item \texttt{file2.tex}
				\end{itemize}
			\end{itemize}
		\end{itemize}
	\column{0.45\textwidth}
		\footnotesize
		\begin{block}{Código [\texttt{main.tex}]}
			\latexfile{codigos/modular-3/main.tex}
		\end{block}
		\begin{block}{Código [\texttt{file1.tex}]}
			\latexfile{codigos/modular-3/folder1/file1.tex}
		\end{block}
		\begin{block}{Código [\texttt{file2.tex}]}
			\latexfile{codigos/modular-3/folder1/folder2/file2.tex}
		\end{block}
	\end{columns}
\end{frame}

\begin{frame}{El paquete \texttt{import}}
	Este paquete nos permite utilizar \texttt{paths} relativos
	\begin{columns}
		\column{0.45\textwidth}
			\begin{itemize}
				\item \texttt{main.tex}
				\item folder1
				\begin{itemize}
					\item \texttt{file1.tex}
					\item folder2
					\begin{itemize}
						\item \texttt{file2.tex}
					\end{itemize}
				\end{itemize}
			\end{itemize}
		\column{0.45\textwidth}
			\footnotesize
			\begin{block}{Código [\texttt{main.tex}]}
				\latexfile{codigos/import/main.tex}
			\end{block}
			\begin{block}{Código [\texttt{file1.tex}]}
				\latexfile{codigos/import/folder1/file1.tex}
			\end{block}
			\begin{block}{Código [\texttt{file2.tex}]}
				\latexfile{codigos/import/folder1/folder2/file2.tex}
			\end{block}
		\end{columns}
\end{frame}