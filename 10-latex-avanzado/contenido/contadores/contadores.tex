% LTeX: language=es-ES

\section{Números y contadores}

\begin{frame}[fragile]{Operaciones con números}
    \latexinline{\numexp} permite operar con enteros. 
    \latexinputandcompile{codigos/numeros.tex}

    \bigskip 

    Se envuelve en \latexinline{\the} y \latexinline{\relax} por razones internas de \LaTeX\footnote{Expansión de tokens que discutiremos más adelante}.

    Se pueden operar también dimensiones, y otros tipos de números. 
    
    \bigskip

    Ver: 

    \url{https://tex.stackexchange.com/questions/245635/formal-syntax-rules-of-dimexpr-numexpr-glueexpr}
\end{frame}

\begin{frame}[fragile]{Contadores}
    Los contadores son enteros con funcionalidades adicionales

    \footnotesize
	\latexinputandpdf{codigos/mover}
\end{frame}

\begin{frame}[fragile]{Contadores}
    \footnotesize
	\latexinputandpdf{codigos/representar}
\end{frame}

\begin{frame}{\texttt{numberwithin}}
    Se puede forzar a un contador a depender de otros. Por ejemplo
    
    \latexinline{\numberwithin{equation}{section}}

    hace que las ecuaciones de la sección $n$ se numeren como ($n.1$), ($n.2$), ...

    \bigskip 

    \pause 

    Esencialmente lo que se hace es reiniciar el contador de ecuaciones cuando se cambia de sección y hacer

    \latexinline{\renewcommand{\theequation}{\thesection.\arabic{equation}}}
\end{frame}