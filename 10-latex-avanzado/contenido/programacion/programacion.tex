% LTeX: language=es-ES

\section{Control de flujo}

\begin{frame}[fragile]{Condicionales nativos}
	\url{https://en.wikibooks.org/wiki/TeX/if}
	\begin{quote}
		The \latexinline{\if} command denotes the start of an if-then-else control structure. The forms <token-1> and <token-2> must expand to tokens. <token-1> and <token-2> can be either a character or a control sequence. If <token-1> and <token-2> both expand to the same character code then <tex-code-1> is processed; otherwise it is ignored. If the \latexinline{\else} section is included and <token-1> and <token-2> expand to different character codes, then <tex-code-2> is processed; otherwise it is ignored. 
	\end{quote}
\end{frame}

\begin{frame}[fragile]
	\footnotesize
	\latexinputandpdf{codigos/if}
\end{frame}

\begin{frame}[fragile]{\latexinline{\ifx}}
	\latexinline{\if} ``expande'' (es decir, evalúa) los tokens antes de compararlos. 

	\bigskip 

	Un comando similar es \latexinline{\ifx}. La diferencia es complicada: \href{https://tex.stackexchange.com/questions/127502/difference-between-if-and-ifx}{StackOverflow}.

	\bigskip 

	Un ejemplo habitual de uso es

	\latexinline{\ifx\mycmd\undefined}
\end{frame}

\begin{frame}[fragile,t]{\latexinline{\ifthenelse}}
	\small

	El paquete \href{https://ctan.org/pkg/ifthen}{\texttt{ifthen}}\
	(mantenido por el equipo de {\LaTeX}) 
	
	nos pertmite un \texttt{if} más ``moderno'':

	\latexinline{\ifthenelse{<expression>}{<true code>}{<false code>}}

	\bigskip

	Las condiciones utilizan
	\begin{table} 
		\centering
	\begin{tabular}{cl|cl}
		\latexinline{==} & equality &
		\latexinline{!=} & inequality \\ 
		\latexinline{<} & less than &
		\latexinline{>} & greater than \\ 
		\latexinline{&&} & AND &
		\latexinline{||} & OR \\
		\latexinline{!} & NOT
	\end{tabular}
	\end{table}

	\footnotesize
	\begin{columns}[] 
	\column{0.45\textwidth}
		Podemos utilizar contadores
		\begin{latexsource}
\newcounter{value}
\setcounter{value}{10}

\ifthenelse{\value{value}>5}
{\color{green}Large value}
{\color{red}Small value}
	\end{latexsource}

	\column{0.45\textwidth}

		E incluso se introduce valores \texttt{bool}
		\begin{latexsource}
\newboolean{draft}
\setboolean{draft}{True}
\ifthenelse{\boolean{draft}}
{Draft version}
{}
		\end{latexsource}
	\end{columns}

	\bigskip 

	\href{https://latexum.com/conditional-expressions-in-latex-a-guide-to-robust-usage/}{Fuente de los ejemplos}

	Ver también \url{https://ctan.org/pkg/xifthen}
\end{frame}

\begin{frame}[fragile]{Bucles}
	El comando \latexinline{\foreach} nos permite hace bucles \texttt{for} 
	
	\latexinlineandcompile{\foreach \x in {1,2,3,0}{\x,}}

	\bigskip 

	Admite notaciones de rango como \latexinline{\{1,2,...,6\}}	

	\bigskip 

	El paquete \href{https://ctan.org/pkg/forloop}{\texttt{forloop}} añade el comando 

	\latexinline{\forloop{counter}{startValue}{condition}{body}}

	\bigskip 

	Ver también el paquete \href{https://ctan.org/pkg/multido}{\texttt{multido}}.
\end{frame}