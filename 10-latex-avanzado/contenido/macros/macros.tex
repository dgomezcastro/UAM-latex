% LTeX: language=es
\section{Macros internos}

\begin{frame}{Comando que almacenan contenido}
    Pensemos un segundo en cómo funciona \latexinline{\author{}}.

    \bigskip
    \begin{itemize}
        \item \latexinline{\author{David Gómez}} almacena el autor del documento.

        \item \latexinline{\maketitle} utiliza esa información en para crear el título.

        \item Con \latexinline{\insertauthor} puedo mostrar el autor del documento.
    \end{itemize}

    Esto se hace con \emph{internal macros} (macros internos)
\end{frame}

\begin{frame}{\emph{Internal macros}}
    Los comandos internos se caracterizan por utilizar la @ al principio del nombre. Para poder utilizar este caracter deberemos convertilo en letra usando \latexinline{\makeatletter} (en inglés @ se lee \emph{at}).

    \medskip

    El siguiente ejemplo crea un comando para fijar una variable (a veces llamado \emph{setter}), y otro para usarla (a veces llamado \emph{getter}).
    \small
    \latexinput{codigos/title}
\end{frame}


