% LTeX: language=es-ES

\usepackage[spanish]{babel}
\addto\captionsspanish{\renewcommand{\partname}{Sesión}}

\newcommand{\firstslide}{
    \frame{\partpage}
}

\usepackage[utf8]{inputenc}
\usepackage{graphicx}
\usepackage{enumerate}
\usepackage{listings}
\usepackage[all]{xy}

\usepackage{tikz}

\mode<presentation>    

% Estilo de diapositiva
\usetheme{Madrid}
\usecolortheme{dolphin}
% Change base colour beamer@blendedblue (originally RGB: 0.2,0.2,0.7)
\colorlet{beamer@blendedblue}{green!40!black} % UAM

\makeatletter
\setbeamertemplate{footline}{
  \leavevmode%
  \hbox{%
  \begin{beamercolorbox}[wd=.4\paperwidth,ht=2.25ex,dp=1ex,center]{author in head/foot}%
    \usebeamerfont{author in head/foot}\insertshortauthor\expandafter\ifblank\expandafter{\beamer@shortinstitute}{}{~~(\insertshortinstitute)}
  \end{beamercolorbox}%
  \begin{beamercolorbox}[wd=.3\paperwidth,ht=2.25ex,dp=1ex,center]{title in head/foot}%
    \usebeamerfont{title in head/foot}\insertshorttitle~(\partname~\thepart)
  \end{beamercolorbox}%
  \begin{beamercolorbox}[wd=.3\paperwidth,ht=2.25ex,dp=1ex,right]{date in head/foot}%
    \usebeamerfont{date in head/foot}\insertshortdate{}\hspace*{2em}
    \insertframenumber{} / \inserttotalframenumber\hspace*{2ex} 
  \end{beamercolorbox}}%
  \vskip0pt%
}
\makeatother

\beamertemplatenavigationsymbolsempty

% Para la parte de los teoremas
\usepackage{amsmath,amsthm}

\theoremstyle{plain}
\newtheorem{teorema}{Teorema}

\usepackage[]{hyperref}

\setbeamerfont{section in toc}{size=\normalsize }
\setbeamerfont{subsection in toc}{size=\small }
\useinnertheme{circles}

% \usepackage{verbatim}
\usepackage{minted}
\newminted[latexsource]{latex}{linenos=false}
\newmintinline[latexinline]{latex}{}
\newmintedfile[latexfile]{latex}{}

\newcommand{\latexinputandcompile}[1]
{
  \begin{columns}
    \column{0.4\textwidth}
    \begin{block}{Código}
        \latexfile{#1}
    \end{block}
    \column{0.4\textwidth}
    \centering
    \fbox{
    \begin{minipage}{\textwidth}
        \input{#1}
    \end{minipage}
    }
  \end{columns}
}

\title{Curso de \LaTeX}
\author{David Gómez-Castro}
\institute[UAM]{Universidad Autónoma de Madrid}
\date{2024-2025}



\begin{document}

\title{Curso de \LaTeX} 
\frame{\titlepage}

\setcounter{part}{-1}
\part{Cuestiones prácticas}
\frame{\partpage}

\begin{frame}{El profesor}
	
\end{frame}

\begin{frame}
	\frametitle{¿Qué es \LaTeX?}
	
	\LaTeX\  es un sistema de preparación de documentos, utilizado en documentos científicos y técnicos. \\[3ex]
	
	\LaTeX\ ¡\textbf{no} es un procesador de textos! 
	
	Nos permite separar el contenido del continente, 
	
	dejando el formato a un lado. \\[3ex]
	
	Por eso, \LaTeX\ se escribe en documentos de texto ``sin formato'' con una cabecera. 
	
	La cabecera dice cómo será el formato (tipo de letra, espaciados, márgenes, títulos...).\\[3ex]
	
\end{frame}

\begin{frame}
	\frametitle{¿Por qué \LaTeX?}	
	\textbf{¿Quién lo usa?} 
	\begin{columns} 
		\column{.3\textwidth}
	\begin{enumerate} 
		\item Las principales revistas del mundo: Nature, Science, PNAS, PLOS, ...
		\item Todas las revistas de Matemáticas
		\item Los profesores en sus apuntes (en la UAM y en todas partes)
	\end{enumerate} 

		\column{.7\textwidth}
	\begin{figure}
		\centering
		\includegraphics[width=\textwidth]{imagenes/perez1}\\
		\hrule ~\\[2ex]
		
		\includegraphics[width=\textwidth]{imagenes/perez2}
	\end{figure}
\end{columns}
\end{frame}

\begin{frame}{Estructura del curso}
	
\end{frame}

\begin{frame}{Procedimiento de evaluación}
	
\end{frame}

\end{document}