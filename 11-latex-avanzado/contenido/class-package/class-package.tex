\section{Creando \texttt{class} y \texttt{package}}

\begin{frame}{Mi propia \texttt{documentclass}}
    \small

    Para crear una \texttt{documentclass} basta crear un archivo llamado \latexinline{<nombre>.cls} e indicar que es una clase. Por ejemplo
    \begin{block}{Código (\texttt{otroarticulo.cls})}
        \latexfile{codigos/otroarticulo.cls}
    \end{block}
    
    \begin{columns}
        \column{0.6\textwidth}
        \begin{block}{Código (\texttt{main.tex})}
            \latexfile{codigos/main.tex}
        \end{block}
        \column{0.25\textwidth}
        \centering
        \fbox{
        \begin{minipage}{\textwidth}
            \includegraphics[width=.8\textwidth]{codigos/main.pdf}
        \end{minipage}
        }
        \texttt{main.pdf}
      \end{columns}
\end{frame}

\begin{frame}[fragile]
    Los archivos \texttt{.cls} suelen ser bastante complejos. 
    
    Por ejemplo, puede encontrarse \texttt{article.cls} escribiendo en terminal: 

    \begin{minted}{sh}
kpsewhich article.cls
    \end{minted}
\end{frame}

\begin{frame}[fragile]{¿Qué puedo hacer en una \texttt{class}?}
    En un \texttt{.csl} podemos:
    \begin{itemize}
        \item Basarnos en clases existentes usando \latexinline{\LoadClass}
        \item Incluir paquetes usando \latexinline{\RequirePackage} 
        \item Crear opciones para el paquete con \latexinline{\DeclareOption}
    \end{itemize} 

    Muchos ejemplos en 
    
    \url{https://www.overleaf.com/learn/latex/Writing_your_own_class}
\end{frame}

\begin{frame}[fragile]{Creando un \texttt{package}}
    Los paquetes se guardan en archivos \texttt{<nombre>.sty}. 
    \begin{block}{Código (\texttt{examplepackage.sty})}
    \small
    \begin{latexsource}
\NeedsTeXFormat{LaTeX2e}
\ProvidesPackage{examplepackage}[2014/08/24 Example LaTeX package]
    \end{latexsource}
    \end{block}

    Leer más: \url{https://www.overleaf.com/learn/latex/Writing_your_own_package}
\end{frame}
