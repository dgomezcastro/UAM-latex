\section{Referencias a bibliografía} 

\begin{frame}
	\frametitle{Bibliografía nativa}
	
		\begin{columns}
		\column{.5\textwidth}
		\begin{block}{Código}
			\tiny
			\latexfile[fontsize=\tiny]{codigos/ejemplo-cite.tex}
		\end{block}
		\column{.5\textwidth}
		\fbox{
			\includegraphics[trim=4cm 15cm 4cm 3cm,clip,   scale=.4]{codigos/ejemplo-cite.pdf}
		}
	\end{columns}
\end{frame}

\begin{frame}
	\frametitle{Bibtex: formato automático} 
	
	\begin{columns}
		\column{.5\textwidth}
		\begin{block}{Código [ejemplo-bibtex.tex]}
			\latexfile[fontsize=\tiny]{codigos/ejemplo-bibtex.tex}
		\end{block}
		\begin{block}{Código [ejemplo-bibliografia.bib]}
			\latexfile[fontsize=\tiny]{codigos/ejemplo-bibliografia.bib}
		\end{block}
	
		\column{.5\textwidth}
		\fbox{
			\includegraphics[trim=4cm 15cm 4cm 3cm,clip,   scale=.4]{codigos/ejemplo-bibtex.pdf}
		}
	
	\vspace{.5cm} 
	{\small 
		\textbf{Observación.}
		Bibtex, al compilar, genera un archivo \texttt{.bbl} que contiene una bibliografía nativa.
	}
	\end{columns}
\end{frame}
