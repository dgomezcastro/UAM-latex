\setcounter{part}{2}
\part{\LaTeX\ como editor de texto avanzado}
\frame{\partpage}

\frame{\tableofcontents}

\label{sesion:3}

\section{Insertando objetos}
\frame{\sectionpage}

\begin{frame}
	\frametitle{Flotantes y su localización} 
	\small 
	Los objetos con los que vamos a trabajar: figuras, tablas, etc... se conocen como flotantes. \\ [2ex] 
	
	Por defecto \LaTeX\  los coloca donde menos moleste: en el lugar del texto donde hemos colocado el código, al principio de la página o al final de la página.  \\ [2ex]
	
	\pause 
	
	Podemos especificar dónde colocarlos mediante parámetros optativos:
	
	\latexinline{\begin{figure}[placement specifier]}. \\ [2ex]
		
	Las opciones son las siguientes
	\begin{table}
		\tiny 
	\begin{tabular}{l|l}
	Specifier	& Permission \\
	h&	Place the float here (approximately at the same point it occurs in the source text)\\
	t&	Position at the top of the page.\\
	b&	Position at the bottom of the page. \\
	p&	Put on a special page for floats only. \\
	!&	Override internal parameters LaTeX uses for determining ``good'' float positions. \\
	H&	Places the float at precisely the location in the LaTeX code. Requires the \latexinline{float} package.
	\end{tabular}
	\end{table} 
	También se admiten cadena \latexinline{htb} significa: inténtalo en su sitio, si no ponlo arriba y, si no, abajo.
\end{frame}

\subsection{Figuras}

\begin{frame}[fragile]
\frametitle{Figuras}
Para incluir imágenes empleamos el paquete \texttt{graphicx}. Si empleamos el compilador \texttt{pdflatex}. Podemos emplear imágenes \texttt{.pdf, .png}. \\[2ex]

\begin{columns}
\column{0.5\textwidth}
Tiene los siguientes parámetros:
\begin{enumerate}
	\item<1-> \alert<1>{\latexinline{<path>}} \only<1>{Es la dirección del archivo a incluir. Lo mejor es escribirla relativa al directorio esto es 
	\footnotesize
	\texttt{<path>=archivo.pdf}}
	\item<2-> \alert<2>{\latexinline{<scale>}} \only<2>{Valor entre 0 y 1 al que escalar la figura.}
	\item<3-> \alert<3>{\latexinline{<caption>}} \only<3>{El contenido de \texttt{<caption>} será el pie de foto, es decir, aparecerá
	Figura x: caption}	
\end{enumerate}
\column{0.4\textwidth}
\begin{block}{Código}
	\tiny
	\verb|\begin{figure}[h!]|
	
	\hspace{1ex} \latexinline{\includegraphics[scale=}~\alert<2>{\latexinline{<scale>]}}~\alert<1>{\latexinline}
	
	\hspace{1ex} \latexinline{\caption}~\alert<3>{\latexinline}
	
	\latexinline{\end{figure}}
\end{block}
\end{columns}
\end{frame}

\begin{frame}[fragile]
    \frametitle{Figura}
    \framesubtitle{Ejemplo}
    \begin{columns}
        \column{.5\textwidth}
        \begin{block}{Código [figura.tex]}
            \latexfile[fontsize=\small]{codigos/figure1.tex}
        \end{block}
        \column{.4\textwidth}
        \centering
        \fbox{
        \includegraphics[trim=7cm 10cm 7cm 10cm,clip,   scale=.5]{codigos/figure1.pdf}
        }
	
	\begin{flushleft}
		Si añadimos el paquete babel, el texto ``Figure'' cambia a ``Figura''
	\end{flushleft}
    \end{columns}
\end{frame}

\subsection{Tablas}
\begin{frame}[fragile]
\frametitle{Tablas}

Las el entorno \latexinline{table} es equivalente a figura, pero al generar el \latexinline{caption} obtendremos Cuadro (como recomienda la RAE). \\ [2ex]

En contenido de la tabla se introduce de manera similar a una matriz
\begin{columns}
	\column{0.4\textwidth}
	\begin{enumerate}
		\item<1-> \texttt{<align>}: 
		\only<1>{
			\begin{enumerate}[a)]
				\item \texttt{l}: izquierda 
				\item \texttt{c}: centrado 
				\item \texttt{r}: derecha
		\end{enumerate}}
		\item<2-> \latexinline{&} \only<2>{Separación entre cuadros en la misma fila}
		\item <3-> \texttt{|} \only<3>{Si se desea línea vertical entre dos columnas}
		\item <4-> \latexinline{\hline} \only<4>{Si se desea una línea horizontal.}
	\end{enumerate}
	
	
	\column{0.5\textwidth}
	\begin{block}{Código}
		\begin{latexsource*}{fontsize=\footnotesize}
\begin{tabular}{<align>| ... }
	cuadro1 & cuadro2 & ... \\
	\hline\\
 	...
\end{tabular}
		\end{latexsource*}
	\end{block}
\end{columns}
\end{frame}

\begin{frame}[fragile]
\frametitle{Tablas}
\framesubtitle{Ejemplo}
	\latexinputandcompile{codigos/table1.tex}
\end{frame}

\begin{frame}{Uso avanzado de tablas}
	\url{https://www.overleaf.com/learn/latex/Tables}
\end{frame}

\begin{frame}{Ejercicio de tablas}

\end{frame}

\begin{frame}{Tablas para vagos}
	Hay múltiples páginas que permiten convertir tablas de excel en latex
	\begin{itemize}
		\item \url{https://www.tablesgenerator.com/}
		\item \url{https://www.latex-tables.com/}
		\item \url{https://tableconvert.com/latex-generator} 
	\end{itemize}
\end{frame}

\subsection{Códigos} 

\begin{frame}{Código}
	Hay diferentes paquetes para incluir código fuente:
	\begin{itemize}
		\item \texttt{verbatim}
		\item \texttt{listings}
		
		MathWorks ofrece un paquete (a descargar \href{https://es.mathworks.com/matlabcentral/fileexchange/8015-m-code-latex-package}{aquí}) para incluir código con formato de matlab

		\item \texttt{minted}. En mi opinión, el más completo. Será el que usemos.
		
		Un tutorial más completo en la \href{https://www.overleaf.com/learn/latex/Code_Highlighting_with_minted}{web de Overleaf}.
	\end{itemize}
\end{frame}

\begin{frame}
\frametitle{Añadir código} 
El paquete \texttt{listing} nos permite introducir código de diferentes lenguajes
\begin{columns} 
	\column{.3\textwidth}
	\begin{block}{Código} 
		\latexfile[fontsize=\footnotesize, firstline=3]{codigos/ejemplo-c.tex}
	\end{block} 
	\column{.6\textwidth}
	\begin{figure} 
		\fbox{\includegraphics[width=\textwidth]{codigos/ejemplo-c.pdf}}
		\caption{Resultado de compilar}
	\end{figure} 
\end{columns}
\end{frame}

\begin{frame}
\frametitle{Añadir código} 
También se puede cargar de un fichero externo
\begin{columns} 
	\column{.4\textwidth}
	\begin{block}{Código [ejemplo-c-2.tex]} %[\texttt{ejemplo-c-2.tex}]
		\latexfile[fontsize=\footnotesize, firstline=3]{codigos/ejemplo-c-2.tex}
	\end{block} 
	\begin{block}{Código [codigo2.cpp]}
		\inputminted[fontsize=\footnotesize]{c++}{codigos/codigo2.cpp}
	\end{block}
	\column{.6\textwidth}
	\begin{figure} 
		\fbox{\includegraphics[width=\textwidth]{codigos/ejemplo-c-2.pdf}}
		\caption{Resultado de compilar \texttt{ejemplo-c-2.tex}}
	\end{figure} 
\end{columns}
\end{frame}

\section{Formato avanzado}
\frame{\sectionpage}
\subsection{Secciones e índices}

\begin{frame}[fragile]
	\frametitle{Secciones e índices} 
	
	Naturalmente \LaTeX\ permite estructurar el documento en secciones. \\ 
	
	\pause 
	
	Para empezar una nueva sección con título basta con indicarlo y dar un título utilizando los siguientes comandos:
	
	\begin{table} 
		\centering
	\begin{tabular}{c|l}
	-1 & \latexinline{\part{titulo}} \\
	0 &	\latexinline{\chapter{titulo}} \\
		1	& \latexinline{\section{titulo}} \\
		2	& \latexinline{\subsection{titulo}} \\
		3	& \latexinline{\subsubsection{titulo}} \\
		4	& \latexinline{\paragraph{titulo}} \\
		5	& \latexinline{\subparagraph{titulo}}
	\end{tabular}
	\end{table} 

	\pause 

	Se puede generar el índice introduciendo \latexinline{\tableofcontents}. Este comando admite parámetros optativos.  \\ 
	
	\pause 
	
	También se pueden hacer otras tablas de contenidos: \latexinline{\listoffigures}, \latexinline {\listoftables}.
\end{frame}



\subsection{El formato de página}

%Geometry, twocolumn...

\begin{frame}[fragile]
\frametitle{El paquete \texttt{fancyhdr}}
Una forma sencilla de darle un buen estilo a nuestro archivo es emplear el paquete \texttt{fancyhdr}. Tiene varias opciones, lo más sencillo es emplear un paquete predefinido
\begin{block}{Código}
	\begin{latexsource}
	\usepackage{fancyhdr}
	\pagestyle{<style>}
	\end{latexsource}
\end{block}

\texttt{<style>}:
\scriptsize
\begin{tabular}{r|l l }
	& Encabezado & Pie de página\\
	\texttt{empty} & Vacío & Vacío\\
	
	\texttt{plain} & Vacío & Número de página centrado\\
	
	\texttt{headings} & Nombre del capítulo y número de página & Vacío\\
	
	\texttt{myheadings} & Número de página y otros & Vacío\\
	\texttt{fancy} & Capítulo y sección & Número de página
\end{tabular}
\end{frame}

\begin{frame}{Márgenes: el paquete \texttt{geometry}}
\url{https://www.overleaf.com/learn/latex/Page_size_and_margins}
\end{frame}
%
%\subsection{El formato de la Facultad}
%
%\begin{frame}
%\frametitle{El formato de la Facultad}
%La Facultad va a incorporar un formato de TFG en el que toda el formato necesario va preparada en la clase \texttt{UCMmatTFG}.
%
%\begin{block}{Código. Uso de  \texttt{UCMmatTFG}}
%\tiny
%\latexfile[fontsize=\tiny,firstline=4,lastline=4]{codigos/UCMmatTFG-prueba.tex}
%[...]
%\latexfile[fontsize=\tiny,firstline=11,lastline=11]{codigos/UCMmatTFG-prueba.tex}
%[...]
%\latexfile[fontsize=\tiny,firstline=30,lastline=45]{codigos/UCMmatTFG-prueba.tex}
%\end{block}
%%	\begin{columns} 
%%		\column{.4\textwidth}
%%			\begin{minipage}[t][.6\textheight][l]{\linewidth}
%%				\begin{block}{Código. Uso de  \texttt{UCMmatTFG}}
%%					\latexfile[fontsize=\tiny,lastline=24]{codigos/UCMmatTFG-prueba.tex}
%%				\end{block}
%%			\end{minipage} 
%%		\column{.4\textwidth}
%%			\begin{minipage}[t][.6\textheight][l]{\linewidth}
%%				\begin{block}{Código. Uso de  \texttt{UCMmatTFG}}
%%					\latexfile[fontsize=\tiny,firstline=25,lastline=43]{codigos/UCMmatTFG-prueba.tex}
%%				\end{block}
%%			\end{minipage}
%%	\end{columns} 
%\end{frame}