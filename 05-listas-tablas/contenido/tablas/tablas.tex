\section{Tablas}
\begin{frame}[fragile]
\frametitle{Tablas}

Las el entorno \latexinline{table} es equivalente a figura, pero al generar el \latexinline{caption} obtendremos Cuadro (como recomienda la RAE). \\ [2ex]

En contenido de la tabla se introduce de manera similar a una matriz
\begin{columns}
	\column{0.4\textwidth}
	\begin{enumerate}
		\item<1-> \texttt{<align>}: 
		\only<1>{ los básicos son
			\begin{enumerate}[a)]
				\item \texttt{l}: izquierda 
				\item \texttt{c}: centrado 
				\item \texttt{r}: derecha
				\item \latexinline{p{<width>}}: justificado a un ancho dado
		\end{enumerate}}
		\item<2-> \latexinline{&} \only<2>{Separación entre cuadros en la misma fila}
		\item <3-> \texttt{|} \only<3>{Si se desea línea vertical entre dos columnas}
		\item <4-> \latexinline{\hline} \only<4>{Si se desea una línea horizontal.}
	\end{enumerate}
	
	
	\column{0.5\textwidth}
	\begin{block}{Código}
		\begin{latexsource*}{fontsize=\footnotesize}
\begin{tabular}{<align>| ... }
	cuadro1 & cuadro2 & ... \\
	\hline\\
		...
\end{tabular}
		\end{latexsource*}
	\end{block}
\end{columns}
\end{frame}

\begin{frame}[fragile]{Ejemplo básico}
	\small
	\latexinputandcompile{codigos/tabla-basica.tex}
\end{frame}

\begin{frame}[fragile]{Más alineaciones}
	\tiny
	Hay paquete especiales que añaden funcionalidades
	\latexinputandpdf{codigos/ancho-columna}
\end{frame}

\begin{frame}[fragile]{Líneas horizontales parciales \texttt{cline}}
	\latexinputandcompile{codigos/cline.tex}	
\end{frame}

\begin{frame}[fragile]{Combinando columnas \texttt{multicolumn}}
	\footnotesize
	\latexinputandpdf{codigos/multicol}
\end{frame}

\begin{frame}[fragile]{Combinando filas \texttt{multirow}}
	\footnotesize 
	Esto requiere cargar el paquete \texttt{multirow}.
	\latexinputandpdf{codigos/multirow}
\end{frame}

\begin{frame}[fragile]{Usando \texttt{multicolumn} y \texttt{multirow}}
	\includegraphics{codigos/multirow-and-multicol.pdf}
\end{frame}

\begin{frame}[fragile]
	\tiny
	\begin{block}{Código}
		\latexfile{codigos/multirow-and-multicol.tex}
    \end{block}
\end{frame}

\begin{frame}{Algunos comentario más sobre tablas}
	Más info: \url{https://www.overleaf.com/learn/latex/Tables}

	\bigskip

	Hay múltiples páginas que permiten generar tablas con más comodidad
	\begin{itemize}
		\item \url{https://www.tablesgenerator.com/}
		\item \url{https://www.latex-tables.com/}
		\item \url{https://tableconvert.com/latex-generator} 
	\end{itemize}
\end{frame}

\begin{frame}{Paquete alternativo: \texttt{booktabs}}

	El paquete \texttt{booktabs} ofrece una alternativa a las tablas convencionales, que algunos autores valoran positivamente:

	\url{https://nhigham.com/2019/11/19/better-latex-tables-with-booktabs/}
	
\end{frame}

\begin{frame}[fragile]{Tablas en ecuaciones}
	El modo matemático incluye \latexinline{array}.

	\latexinputandcompile{codigos/array.tex}

	\bigskip

	En general, para una función definda a trazos se recomienda utilizar el entorno
	\latexinline{cases} (si no hay fracciones), o \latexinline{dcases} del paquete \latexinline{mathtools} (si hay fracciones).
\end{frame}