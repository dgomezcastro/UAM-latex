\subsection{Tablas}
\begin{frame}[fragile]
	\frametitle{Tablas}

	Las el entorno \latexinline{table} es equivalente a figura, pero al generar el \latexinline{caption} obtendremos Cuadro (como recomienda la RAE). \\ [2ex]
	
	En contenido de la tabla se introduce de manera similar a una matriz
	\begin{columns}
		\column{0.4\textwidth}
		\begin{enumerate}
			\item<1-> \texttt{<align>}: 
			\only<1>{
				\begin{enumerate}[a)]
					\item \texttt{l}: izquierda 
					\item \texttt{c}: centrado 
					\item \texttt{r}: derecha
			\end{enumerate}}
			\item<2-> \latexinline{&} \only<2>{Separación entre cuadros en la misma fila}
			\item <3-> \texttt{|} \only<3>{Si se desea línea vertical entre dos columnas}
			\item <4-> \latexinline{\hline} \only<4>{Si se desea una línea horizontal.}
		\end{enumerate}
		
		
		\column{0.5\textwidth}
		\begin{block}{Código}
			\begin{latexsource*}{fontsize=\footnotesize}
	\begin{tabular}{<align>| ... }
		cuadro1 & cuadro2 & ... \\
		\hline\\
		 ...
	\end{tabular}
			\end{latexsource*}
		\end{block}
	\end{columns}
	\end{frame}
	
	\begin{frame}[fragile]
	\frametitle{Tablas}
	\framesubtitle{Ejemplo}
		\latexinputandcompile{codigos/table1.tex}
	\end{frame}
	
	\begin{frame}{Uso avanzado de tablas}
		\url{https://www.overleaf.com/learn/latex/Tables}
	\end{frame}
	
	\begin{frame}{Ejercicio de tablas}
	
	\end{frame}
	
	\begin{frame}{Tablas para vagos}
		Hay múltiples páginas que permiten convertir tablas de excel en latex
		\begin{itemize}
			\item \url{https://www.tablesgenerator.com/}
			\item \url{https://www.latex-tables.com/}
			\item \url{https://tableconvert.com/latex-generator} 
		\end{itemize}
	\end{frame}