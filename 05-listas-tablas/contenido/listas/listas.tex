\section{Listas}

\begin{frame}[fragile]{Listas básicas}
    \latexinputandpdf{codigos/ejemplo-basico}
\end{frame}

\begin{frame}{Cambiando el comportamiento por defecto}
    El comportamiento por defecto es

    \texttt{enumerate} es números, y para \texttt{itemize} es $\bullet$.

    \bigskip

    El comportamiento de \texttt{itemize se puede modificar}
    \latexinline{\renewcommand{\labelitemi}{$\cdot$}}

    Además, \latexinline{\item} admite como entrada optativa el texto que queramos poner, p.e. \latexinline{\item[!]}.

    \bigskip

    Para configurar \texttt{enumerate} es preferible usar uno (y sólo uno) de los paquetes:

    \texttt{enumitem} (más completo, y recomendado) o \texttt{enumerate} (más sencillo).
\end{frame}

\begin{frame}[fragile]{El paquete \texttt{enumerate}}
    La documentación en CTAN (4 páginas) es muy clara:
    \begin{quote}
        This package gives the enumerate environment an optional argument which determines the style in which the counter is printed.

        An occurrence of one of the tokens A a I i 1 produces the value of the counter printed with (respectively) \latexinline{\Alph \alph \Roman \roman} or \latexinline{\arabic}.

        These letters may be surrounded by any strings involving any other TEX expressions, however the tokens A a I i 1 must be inside a \latexinline{{ }} group if they are not to be taken as special.
    \end{quote}
\end{frame}

\begin{frame}[fragile]
    \footnotesize
    \latexinputandpdf{codigos/enumerate}
\end{frame}

\begin{frame}[fragile]{Introducción a contadores}
    En la documentación anterior habla de \latexinline{\Alph \alph \Roman \roman} or \latexinline{\arabic}.

    \bigskip

    Esto son términos de contadores que veremos más adelante. Baste, por ahora, un pequeño ejemplo:

    \latexinputandcompile{codigos/contadores.tex}

\end{frame}

\begin{frame}[fragile]{El paquete \texttt{enumitem}}
La documentación de este paquete en \href{https://ctan.org/pkg/enumitem}{CTAN} tiene 23 páginas.

\bigskip

Tiene varias funcionalidades muy completas:
\begin{enumerate}
\item Control de márgenes y espaciado

\item Control de etiquetas muy detallada (aunque algo farragoso). Por ejemplo

\latexinline{\begin{enumerate}[label=\emph{\alph*}), ref=\emph{\alph*}]}

\item Permite continuar un la numeración de un \texttt{enumerate} anterior, con la sintaxis

\latexinline{\begin{enumerate}[resume]}.

    \item Permite crear nuevos tipos de listas, con la sintaxis:

          \latexinline{\newlist{<name>}{<type>}{<max-depth>}}
\end{enumerate}

\item Utiliza los contadores \latexinline{enumi}, \latexinline{enumii}, \latexinline{enumiii} que se puede llamar con \latexinline{\theenumi}, ...
\end{frame}


% \begin{frame}{Listas anidadas}
%     %TODO

% \end{frame}