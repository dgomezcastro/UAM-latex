%\documentclass[draftmode]{clases/UCMmatTFG}

% Para la version final comentar lo anterior y descomentar esto:
\documentclass{clases/UCMmatTFG}

%% *************** Definicion de entornos teorema *************
% Las definiciones de Teorema, Proposicion, Lema, Definicion, ...
% se encuentran en el siguiente archivo
\input{clases/def-teoremas}

\begin{document}

%% ****************** Informacion del trabajo ********************
% Thesis title and author information, refernce file for biblatex
\input{tfg-info}

%% ************************* Portada ***************************
\makeTFGtitle

%% ********************* Agradecimientos **********************
\input{agradecimientos/agradecimientos}

\cleardoublepage
%%************************ Resumen ***************************
\input{resumen/resumen}

%% ************************* Indice *****************************
\TFGtableofcontents

%% ************************* Capitulos *************************
%% Se recomienda escribir cada capitulo en un archivo distinto
%% para evitar grandes tamano de archivos


\pagestyle{cuerpo-tfg} 	%Cambiar formato de pagina
 
% Introducimos el primer capitulo
\section{Ecosistemas}

\begin{frame}[fragile]{Ecosistemas}
    Hay dos ecosistemas populares para crear gráficos directamente en \LaTeX :
    \begin{itemize}
        \item PGF TikZ: \href{https://github.com/pgf-tikz/pgf?tab=readme-ov-file}{GitHub} y \href{https://ctan.org/topic/pgf-tikz}{CTAN}
        \item PSTricks: \href{https://tug.org/PSTricks/main.cgi}{web oficial} y \href{https://ctan.org/topic/pstricks}{CTAN}
    \end{itemize}

    En este curso utilizaremos PGF Tikz, que es el más popular.
\end{frame}

\begin{frame}{PGF Tikz}
    
\end{frame}

% Capitulo 2
\input{cap2/cap2}

% Capitulo 3
%\input{cap3/cap3}

%% OBSERVACION
% Para el correcto funcionamiento de caracteres especiales: acento, ñ
% el archivo .tex debe estar codificado con UTF-8.
% Para crear un capitulo nuevo se sugiere duplicar (copiar y pegar con distinto nombre)
% un capitulo existente

% Conclusiones
\input{conclusiones/conclusiones}


%% ************************ Bibliografia ********************
%% Crea la bibliografia utilizando bibtex
%% Carga la informacion del archivo bibliografia/referencias.bib
\input{clases/bibliografia}


%% ************************* Apendices ***********************
%% Todo que aparezca a partir del comando \appendix es numerado
%% con letras. Incluir apendices relevantes en archivos separados
\appendix 

\input{apendice1/apendice1}
\end{document}

