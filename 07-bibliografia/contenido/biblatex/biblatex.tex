\section{\texttt{biblatex}}

\begin{frame}{\texttt{biblatex}}
	Los archivos \texttt{.bst} son una pequeña tortura, y no permiten hacer modificaciones menores del estilo sin meterse ``en todo el tomate''.

    \bigskip

    \url{https://ctan.org/pkg/biblatex}:
    \begin{quote}
        BibLaTeX is a complete reimplementation of the bibliographic facilities provided by LaTeX. Formatting of the bibliography is entirely controlled by LaTeX macros, and a working knowledge of LaTeX should be sufficient to design new bibliography and citation styles.
    \end{quote}

    \texttt{biblatex} también utiliza archivos \texttt{.bib}, pero nos da más flexibilidad de uso.
    
\end{frame}

\begin{frame}{\texttt{biblatex}: ejemplo básico}
    \footnotesize
    \latexinputandpdf{codigos/ejemplo-biblatex}
\end{frame}

\begin{frame}{\texttt{biblatex}: ejemplo con algunas configuraciones}
    \scriptsize
    \latexinputandpdf{codigos/ejemplo-biblatex-advanced}

    \bigskip 

    Más estilos en: 
    \url{https://www.overleaf.com/learn/latex/Biblatex_citation_styles}
\end{frame}


\begin{frame}[fragile]{\texttt{biber}}
    \small 

    \texttt{biblatex} puede utilizar \texttt{bibtex} para generar la bibliografía.
    
    \bigskip 
    
    \url{https://ctan.org/pkg/biblatex}:
    \begin{quote}
        BibLaTeX uses its own data backend program called “biber” to read and process the bibliographic data. With biber, BibLaTeX has many features rivalling or surpassing other bibliography systems.
    \end{quote}

    \bigskip 


    \url{https://en.wikipedia.org/wiki/Biber_(LaTeX)}
    
    \begin{quote}
        Biber is a bibliography information processing program that works in conjunction with the LaTeX package BibLaTeX and offers full Unicode support.
    
        Biber is a widely used replacement for the BibTeX software. Both generate a bibliography in LaTeX, but Biber offers a large superset of BibTeX functionality. It also offers full Unicode support, which is hard to achieve with BibTeX. Given the same data file as input, biber should output a functionally identical .bbl file as BibTeX.
    \end{quote}

    Se puede elegir uno u otro con 
    
    \latexinline{\usepackage[backend=biber]{biblatex}}
\end{frame}