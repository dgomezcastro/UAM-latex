\section{\texttt{bibtex}}

\begin{frame}[fragile]{\texttt{bibtex}: formato automático}
    \url{https://en.wikipedia.org/wiki/BibTeX}:
    \begin{quote}
        BibTeX is both a bibliographic flat-file database file format and a software program for processing these files to produce lists of references (citations). The BibTeX file format is a widely used standard with broad support by reference management software.
    \end{quote}

    Esta herramienta permite guardar la bibliografía en una ``basada de datos'' y elegir el formato por separado.

    \bigskip 

    Algunas mejoras importantes:
    \begin{enumerate}
        \item Formato automático y fácil de cambiar
        \item Ordenación automática
        \item Sólo aparecen las referencias usadas\footnote{aunque se puede especificar que aparezcan todas}
    \end{enumerate}

    
\end{frame}

\begin{frame}[fragile]
	\frametitle{\texttt{bibtex}: ejemplo} 
	
	\begin{columns}
		\column{.5\textwidth}
		\begin{block}{Código [ejemplo-bibtex.tex]}
			\latexfile[fontsize=\tiny]{codigos/ejemplo-bibtex.tex}
		\end{block}
		\begin{block}{Código [ejemplo-bibliografia.bib]}
			\latexfile[fontsize=\tiny]{codigos/ejemplo-bibliografia.bib}
		\end{block}
	
		\column{.5\textwidth}
		\fbox{
			\includegraphics[width=\textwidth]{codigos/ejemplo-bibtex.pdf}
		}
	\end{columns}
\end{frame}

\begin{frame}{Cómo funciona \texttt{bibtex}}
    \begin{figure}
        \includegraphics[width=.9\textwidth]{figuras/LaTeXCompanion-Fig12-1}
        \caption{Tomado de \parencite{mittelbach2004latexcompanion}}
    \end{figure}
\end{frame}

\begin{frame}[fragile]{Cómo funciona \texttt{bibtex} en la terminal}
    \begin{columns}
        \footnotesize
		\column{.45\textwidth}
		Ejecutando
        \begin{minted}{bash}
$ pdflatex ejemplo-bibtex.tex  
        \end{minted}
        Recibimos el error 
        \begin{minted}[breaklines]{bash}
No file ejemplo-bibtex.aux.
[...]
LaTeX Warning: Citation 'einstein' on page 1 undefined on input line 4.
[...]
No file ejemplo-bibtex.bbl.
[...]
LaTeX Warning: There were undefined references.

        \end{minted}
		\column{.45\textwidth}
        \centering
		\fbox{
			\includegraphics[width=.4\textwidth]{codigos/ejemplo-bibtex-1.pdf}
		}
	\end{columns}

    \bigskip 

    Este último aviso también ocurre cuando usamos \latexinline{\ref} con una etiqueta errónea.

\end{frame}

\begin{frame}[fragile]
    \begin{columns}
        \footnotesize
		\column{.45\textwidth}
            Debemos utilizar la herramienta de terminal
		    \begin{minted}[breaklines]{bash}
$ bibtex ejemplo-bibtex                     
            \end{minted}
            Genera el archivo \texttt{ejemplo-bibtex.bbl}

	
		\column{.45\textwidth}
        \begin{block}{Código [ejemplo-bibliografia.bbl]}
			\latexfile[fontsize=\tiny]{codigos/ejemplo-bibtex.bbl-save}
		\end{block}
	\end{columns}

    \begin{columns}
        \footnotesize
		\column{.45\textwidth}
            La siguiente ejecución de \texttt{pdflatex} volverá a dar error, y la última funcionará. Así, ejecutamos 
            \begin{minted}{bash}
$ pdflatex ejemplo-bibtex.tex                     
$ pdflatex ejemplo-bibtex.tex                     
            \end{minted}

	
		\column{.45\textwidth}
		\fbox{
			\includegraphics[width=\textwidth]{codigos/ejemplo-bibtex-3.pdf}
		}
	\end{columns}

    \bigskip

    Como es natural, en nuestra IDE (VSCode, Overleaf, ...) esto ocurre de forma automática. 
    
    Algunos editores pueden requerir pasos adicionales.
\end{frame}

\begin{frame}[fragile]{Los archivos \texttt{.bib}}
    Hay diferentes tipos de entradas: \latexinline{@article, @book, ...}.

    \bigskip    

    Una lista detallada en 

    \url{https://en.wikibooks.org/wiki/LaTeX/Bibliography_Management#BibTeX}

\end{frame}

\begin{frame}[fragile]{\texttt{bibtex}: Estilos}
    Los estilos incluídos con bibtex se pueden encontrar en:

    {\small
    \url{https://www.overleaf.com/learn/latex/Bibtex_bibliography_styles} 
    }

    o 
    
    \parencite[p. 791]{mittelbach2004latexcompanion}

    \bigskip

    Estos estilos vienen definidos por archivos \texttt{.bst}, como por ejemplo

    {\small
    \url{https://www.ctan.org/tex-archive/biblio/bibtex/base/abbrv.bst}
    }

    \bigskip

    Si queremos hacer un estilo propio, deberemos escribir un archivo \texttt{.bst} e importarlo 
    
    (por ejemplo \texttt{./mibibliografia.bst} y \latexinline{\bibliographystyle{mibibliografia}})
\end{frame}